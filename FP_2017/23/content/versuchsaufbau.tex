\section{Versuchsaufbau}
\label{sec:Versuchaufbau}

Der Versuchsaufbau verändert sich bei jedem Versuchsteil. Generell wird für die Versuche eine V-förmige Schiene verwendet, an deren Enden ein Mikrophon und ein Lautsprecher installiert sind. Das Mikrophon ist auf der Schiene freibeweglich, sodass  dieses auf jeden neuen Aufbau angepasst werden kann. Sowohl das Mikrophon- als auch das Lautsprechersignal werden über ein T-Stück gesplittet und mit der Soundkarte des Laptops und einem Oszilloskop verbunden.
Der Laptop wird zur Aufzeichnung der Daten verwendet und das Oszilloskop für die Überprüfung der Resonanzfrequenzen. Für die einzelnen Versuchsteile liegen noch die folgenden Utensilien vor:

\begin{itemize}
\item $\SI{12.5}{\milli\meter}$, $\SI{50}{\milli\meter}$, $\SI{75}{\milli\meter}$ Röhren mit einem Innendurchmesser von jeweils $\SI{25.4}{\milli\meter}$
\item Iriden mit einem Durchmesser von 10, 13 und $\SI{16}{\milli\meter}$
\end{itemize}

Für die Versuche zu Kapitel \ref{sec:Modellierung eines Wasserstoffatoms mit einem sphärischen Resonator} wird zusätzlich ein Kugelresonator verwendet. In der unteren Halbkugel ist ein Lautsprächer und in der oberen ein Mikrophon installiert, die wie zu zuvor mit dem Laptop und dem Oszilloskop verbunden werden. Die obere- und die untere Halbkugel sind relativ zueinander rotierbar, jedoch ist zu beachten, dass der außen angegebene Winkel $\alpha$ nicht dem Polarwinkel $\theta$ entspricht. Die Umrechnung geschieht über die Beziehung
\begin{align}
  	\theta = \text{arccos}(1/2\cos{(\alpha)}-1/2)\;.
    \label{eq:ThetaVonAlpha}
\end{align}
