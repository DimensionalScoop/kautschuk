\section{Zusammenfassung}
\label{sec:Diskussion}
Die Diskussion ist aufgrund des Umfangs des Versuchs bereits größtenteils in den jeweiligen Kapiteln in der Auswertung erfolgt. Zusammenfassend lässt sich festhalten, dass der Versuch die theoretischen Vorhersagen weitestgehend bestätigen kann. Fehler oder Abweichungen sind kaum aufgetreten, da das gesamte System gut abgestimmt ist. Beispielsweise wird der Wert der Schallgeschwindigkeit sehr genau getroffen mit einer geringen Abweichung von $\input{build/RelFehler_c.tex}$. Für die Justage ist es wichtig, die Dämpfung auf einen geeigneten Wert einzustellen, um eine Aussage über die wahren Amplituden treffen zu können. \\
Einige Messergebnisse verbleiben in ihrer Interpretation nicht klar. Zu nennen sind hier:
\begin{itemize}
  \item Für den Polarplot in Abbildung \ref{fig:2_14} ist keine Zuordnung zu $l$ möglich.
  \item Ein Peak wird vermisst in dem ersten Band in Abbildung \ref{fig:3_123}.
  \item Wieso wandert in Abbildung \ref{fig:3_123} der Defektzustand zu kleineren Frequenzen aufgrund einer anderen Position des Defekts?
  \item Wie genau lässt sich für Abbildung \ref{fig:3_110} aus den Spektren einzelner Teilstücke die Überlagerung für eine alternierende Superstruktur ableiten?
\end{itemize}
Schwierigkeiten treten zudem bei der Auflösung der Peaks auf, insbesondere bei schmaleren Bändern im Fall von kleineren Irisdurchmessern. Dies ist ein allgemeines Problem in der Spektoskopie, denn jeder Peak besitzt eine endliche Breite und so führt eine Überlagerung unter Umständen zu Informationsverlust. Hier ist prinzipiell ein numerischer Fit durchzuführen, statt des für die Auswertung verwendeten simplen \emph{peakdetect}-Algorithmus.
