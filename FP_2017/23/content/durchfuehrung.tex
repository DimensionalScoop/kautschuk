\section{Durchführung}
\label{sec:Durchführung}

\subsection{Analogie zwischen einer stehenden Welle und einem quantenmechanischen Teilchen im Rechteckpotential}



\begin{enumerate}
\item Zu Beginn wird ein Übersichtspektrum von $100-10000\si{\hertz}$ einer Röhre, bestehend aus 8$\times$$\SI{75}{\milli\meter}$ Röhren aufgenommen. Die Messung erfolgt in $10\si{\hertz}$ Schritten und einer Zeit von $50\si{\milli\second}$ pro Schritt.
\item Als nächstes wird das Spektrum von $5000-14000\si{\hertz}$, in $5\si{\hertz}$ Schritten und $50$ms pro Schritt, einer 2$\times$$\SI{75}{\milli\meter}$ Röhren gemessen.
\end{enumerate}

\subsection{Modellierung eines Wasserstoffatoms mit einem Kugelresonator}

\begin{enumerate}
\item Das Programm "SpektrumSLC.exe" wird gestartet und der Winkel $\alpha=90$° an dem Resonator eingestellt. In  $\alpha=30$° Schritten werden nun Spektren von $100-10000\si{\hertz}$ aufgenommen.
\item Der Startwert beträgt diesmal $\alpha=0$° und es soll ein Doppelpeak in der Nähe von $5000\si{\hertz}$ in kleinen Schritten untersucht werden. Die Messung wird für $\alpha=20$° und $\alpha=40$° wiederholt.
\item Verschiedene Resonanzen werden gemessen und in einem Polarplot dargestellt, indem die Resonanzfrequenz eingestellt und die stehende Welle im sphärischen Resonator in $\alpha=10$° Schritten vermessen wird.
\end{enumerate}

\subsection{Modellierung eines eindimensionalen Festkörpers}
\label{sec:Durchf3}
\begin{enumerate}
\item Die Bandlücken zwischen $6-9\si{\kilo\hertz}$ werden für Röhren mit 1$\times$$\SI{75}{\milli\meter}$ bis zu 8$\times$$\SI{75}{\milli\meter}$ Teilstücken aufgezeichnet.
\item Aufnahme eines Übersichtsspektrums einer Röhre bestehend aus 12$\times$$\SI{50}{\milli\meter}$ Röhren.
\item In dem gleichen Frequenzbereich wie in der vorherigen Messung soll ein Spektrum aufgenommen werden von 8$\times$$\SI{50}{\milli\meter}$ Röhren die durch $10$, $13$ und $\SI{16}{\milli\meter}$ Iriden separiert werden.
\item Vermessung zweier Röhren, bestehend aus 12$\times$ und 10$\times$$\SI{50}{\milli\meter}$ Teilstücken, getrennt durch $\SI{16}{\milli\meter}$ Iriden.
\item Vermessung einer 8$\times$$\SI{50}{\milli\meter}$ Röhre getrennt durch $\SI{16}{\milli\meter}$ Iriden.
\end{enumerate}

\subsection{Modellierung eines Atoms und Moleküls}
\begin{enumerate}
\item Aufnahme des Spektrums einer $\SI{50}{\milli\meter}$ und einer $\SI{75}{\milli\meter}$ Röhre von $100-22000\si{\hertz}$.
\item Vermessung einer 2$\times$$\SI{50}{\milli\meter}$ Röhren, getrennt durch $10$, $13$ und $\SI{16}{\milli\meter}$ Iriden.
\item Beobachtung der Veränderung der Bandstruktur durch Vermessung von 3$\times$, 4$\times$ und 6$\times$$\SI{50}{\milli\meter}$ Röhren jeweils getrennt durch $10$, $13$ und $\SI{16}{\milli\meter}$ Iriden.
\item Aufzeichnung der Bandlücken bei der Messung einer 5$\times$$\SI{50}{\milli\meter}$ Röhre, abwechselnd getrennt durch $13$ und $\SI{16}{\milli\meter}$ Iriden.
\item Vermessung einer Superstruktur. 5 Einheitszellen, bestehend aus einer $\SI{50}{\milli\meter}$ Röhre mit $\SI{16}{\milli\meter}$ Iris sowie einer $\SI{75}{\milli\meter}$ Röhre mit $\SI{16}{\milli\meter}$ Iris, werden aneinander gelegt.
\item Aufbau einer 12$\times$$\SI{50}{\milli\meter}$ Röhre, die jeweils durch $\SI{16}{\milli\meter}$ Iriden getrennt sind. Vermessung des Spektrums nach dem Hinzufügen einer Störstelle in Form einer $75 $Röhre an den Position 1,3 und 7. Die Positionen sind so definiert, dass das erste Element nach dem Mikrophon sich an Position 1 befindet, das zweite Element aus Sicht des Mikrophons auf Position 2 u.s.w. Ferner wird versuchsweise eine Störstelle in Form einer $\SI{12.5}{\milli\meter}$ Röhre verwendet.
\end{enumerate}

Zu beachten ist, dass gelegentlich zwischen den Messungen die Dämpfung des Signals angepasst werden muss.
