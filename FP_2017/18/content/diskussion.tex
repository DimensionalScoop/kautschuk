\section{Diskussion}
\label{sec:Diskussion}

In Kapitel $\ref{sec:Energiekalibrierung}$ wurde die Energiekalibrierung durchgeführt, indem die Kanalnummern der Maxima gegebenen Werten aus der Literatur zugeordnet wurden. Jedoch gibt es mehr Messwerte in der Literatur als Maxima vorhanden sind. Dabei ist es grundsätzlich möglich, dass eine falsche Zuordnung stattgefunden hat. Es ist jedoch davon auszugehen, dass die Zuordnung korrekt erfolgt ist aufgrund der sehr kleinen Abweichung in Abbildung $\ref{fig:Messdaten_1_rohdaten}$, bzw. den Fitparametern in Gleichung \eqref{eq:Energiefkt}. \\
Die hohe Abweichung in Kapitel $\ref{sec:Photopeak}$ der theoretischen und der gemessenen Halbwertsbreite des Photopeaks von $\input{build/Vergleich_halbwertsbreiten_photo.tex}$ kann nicht eindeutig erklärt werden. Es besteht die Möglichkeit, dass der Fanofaktor in Gleichung \eqref{eq:fano_halbwertsbreite} mit  $F\approx0,1$ \cite{skript} falsch angesetzt wurde. Zu erwarten ist eine kleine Abweichung, da der Photopeak sehr gut durch eine Gaußfunktion in Abbildung angenähert wird, zudem geht hier die Umrechnung von Kanalnummer in Energie ein, die wie zuvor erläutert auch einen kleinen Fehler aufweist. \\
In dem folgenden Kapitel ist zu beachten, das bei der Festsetzung der Kanalnummer der Comptonkante ein großer Fehler anzugeben ist, da der Bereich um die Comptonkante mit keiner Funktion genährt werden konnte aufgrund von starken Fluktuationen, siehe Abbildung $\ref{fig:Cs_probe_Comptonkante}$. Dies ist ebenfalls beim Rückstreupeak zu beachten, trotzdem ist bei der Comptonkante nur eine Abweichung von $\input{build/E_Comptonkante_prozent.tex}$ und bei dem Rückstreupeak eine Abweichung von $\input{build/Compton_ruck_abw.tex}$ zu verzeichnen. Durch Verwendung des indirekten Verfahrens in Kapitel \ref{sec:Compton-Kontinuum} kann die Abweichung auf  $\input{build/Compton_ruck_abw_2.tex}$ reduziert werden.\\
Die hohe Abweichung von $\input{build/Z_Q_wahr.tex}$  in Kapitel $\ref{sec:Absorptionswahrscheinlichkeit}$ zwischen den Verhältnissen der Effizienz und dem Inhalt des Photo- und des Comptonkontinuums kann dadurch erklärt werden, dass der Photopeak nicht nur durch den Photoeffekt entsteht, sondern auch mehrere Comptonstreuungen beteiligt sind. Aus diesem Grund wird anstatt Photopeak auch oft der Begriff Vollenergiepeak verwendet,um diesen Sachverhalt zu unterstreichen.\\
Im Kapitel "Barium-Messung" wird eine Aktivität von $A_\textrm{Ba} = \input{build/Akt_Barium_Mittelw.tex}$ berechnet. Auf Grund eines fehlenden Vergleichswert kann keine direkte Aussage über die Genauigkeit dieses Ergebnisses getroffen werden. Der Vergleich mit der Europium-Probe aus Kapitel \ref{sec:Effizienzbestimmung} zeigt, dass die beiden Aktivitäten in der gleichen Größenordnung liegen.\\
Bei der Messung des Spektrums eines Steins mit unbekannter Zusammensetzung in Kapitel $\ref{sec:Barium_Messung}$ kann nach der Auswertung mit Sicherheit gesagt werden, dass dieser momentan aus Blei und Bismut besteht, da die Abweichungen zu den Literaturwerten in Tabellen \ref{table:Vegleich_Pb} und \ref{table:Vegleich_Bi} kleiner als $1\%$ sind. Weiter wird angenommen, dass der Stein Thorium und Radium enthält, obwohl der Vergleich bei beiden Materialien mit dem Literaturwert eine höhere Abweichung zeigt als bei Blei und Bismut. Mit Hilfe der Zerfallsreihe von Uran in \cite{skript} kann somit die Aussage getroffen werden, dass der Stein ursprünglich teilweise aus Uran bestand.
