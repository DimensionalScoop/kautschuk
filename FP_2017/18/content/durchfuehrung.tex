\section{Durchführung}
\label{sec:Durchführung}
Um die Rohdaten des Detektors verwerten zu können, muss zunächst eine Kalibrierung stattfinden. Dazu wird ein Strahler mit bekannten Photonenenergien benötigt. Es wird \ce{^{152} Eu} verwendet, da dieses Material gleich mehrere verschiedene Linien zeigt. Ziel ist es, die Zuordnung zwischen Kanalnummer und Energiewert herauszufinden. Ferner wird diese Messung auch dazu benötigt, um die Effizienz des Detektors zu bestimmen (siehe Kapitel \ref{sec:effizienz}). Für alle Messungen wird eine Zeit von ca. einer Stunde angesetzt. Prinzipiell sind lange Messungen sinnvoll, um dem statistischen Charakter der $\gamma$-Strahlung sowie deren Detektion Rechnung zu tragen.\\
Des weiteren werden die Spektren von \ce{^{137} Cs}, \ce{^{133} Ba} sowie das Spektrum eines unbekannten Steines aufgenommen.
