\section{Einführung}
Dieses Modul definiert Funktionen, welche die Generierung von \emph{.tex} Dokumenten aus dem internen Python Code heraus ermöglichen sollen. Insbesondere wird ein Augenmerk auf die Generierung von Tabellen gelegt. Grundlage ist die Funktion \emph{make\_table}, welche Daten inklusive Fehlergrößen aufnimmt und einen tex-formatierten string zurückgibt. Falls nun mehrere solcher Tabellen zusammengefügt werden sollen, so kann dies über \emph{make\_composed\_table} geschehen. Sodann ist es zur Zeit vorgesehen, den return Wert dieser beiden Funktionen in eine .tex Datei abzuspeichern. Dies geschieht mit \emph{write}. Es empfiehlt sich, hierfür einen /build Ordner zu verwenden!

Die Funktion \emph{make\_full\_table} übernimmt dann den Teil, aus der generierten Tabelle -- welche momentan aus Zahlenwerten, getrennt durch \& -Zeichen, besteht -- automatisch eine standardisierte, vollständige .tex Tabelle zu erstellen. Der Rückgabewert dieser Funktion ist vom Typ string und startet stets mit "\\begin{table}". Dieser Rückgabewert sollte erneut über \emph{write} in eine .tex File gespeichert werden. Nun ist in den entsprechenden tex Kapiteln nur noch ein "\\input{build/filename.tex}" einzufügen.
