\section{Funktionen}
\subsection{write}
\subsection{make\_SI}
\subsection{make\_table}
\subsection{make\_composed\_table}
\subsection{make\_full\_table}
Call: \t make\_full\_table (caption, label, source\_table, stacking, units)
\begin{table}
  \centering
  \begin{tabular}{l l p{0.7\textwidth}}
    caption   & [string]  & Set the caption of your tabel, displayed usually above the table\\
    label     & [string]  & Set the label used to reference your table in tex code\\
    source\_table & [string]  & Specify the path/filename of the previously generated .tex table (generated by \emph{make\_table}) \\
    stacking  & [list of type int]  & Specifiy which columns are error related values. Here you chose the column(s) (starting from 0) that you will finally see at the table head line. See the examples for fast understanding.\\
    units     & [list of type string] & Here you put in the descriptions of the head line, ordered in a list according to the columns.\\
  \end{tabular}
\end{table}
