\section{Diskussion}
\label{sec:Diskussion}

Die ermittelte Ionisationsenergie des Quecksilbers,
\begin{align*}
  E_{\text{Ion}} &= \input{build/c_ion.tex},
\end{align*}
weist im Vergleich zum Literaturwert \cite{ionisationsenergie},
\begin{align*}
  E_{\text{Ion, Lit}} &= \SI{10.438}{\electronvolt},
\end{align*}
eine relative Abweichung von
\begin{align*}
  E_{\text{Ion,rel}} &= \input{build/c_ion_rel_err.tex}
\end{align*}
auf.
Dies ist ein Indiz dafür, dass das bestimmte Kontaktpotential von etwa
\begin{align*}
  K &= \SI{3.1}{\volt}
\end{align*}
zutreffend ist.\\
Im Bezug zur aufgenommenen Franck-Hertz-Kurve stellt sich die Frage, welche Ursache der unerwartete Verlauf nach dem vierten Maximum hat.
Da Aufnahmen bei verschiedenen Temperaturen zum gleichen Phänomen führten, liegt die Ursache vermutlich in einem systematischen Fehler des Aufbaus noch ungeklärter Herkunft.
Die eindeutig messbaren Maxima, sowie deren Abstände,
\begin{align*}
  \Delta U &= \input{build/b_U_max_delta.tex},
\end{align*}
kommen dem Literaturwert \cite{anregungsenergie},
\begin{align*}
  \Delta U_{\text{Lit}} &= \SI{4,9}{\electronvolt},
\end{align*}
mit einer relativen Abweichung von
\begin{align*}
  \Delta U_{\text{rel}} &= \input{build/b_anregung_rel.tex}
\end{align*}
sehr nahe.
Die elastischen Stöße, selbst wenn sie zentral sind, sind doch statistisch gleichmäßig verteilt und sorgen demnach nur für eine Verbreiterung der Peaks.
Daher können sie vernachlässigt werden.\\
Insgesamt unterstützt der Versuch eine quantenmechanische Sichtweise auf die Elektronenhülle, zumindest bei Quecksilberatomen.
