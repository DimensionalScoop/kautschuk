\section{Auswertung}
\label{sec:Auswertung}
Sämtliche im folgenden durchgeführten Ausgleichsrechnungen werden mit der \emph{curve fit} Funktion aus dem für \emph{Python} geschriebenen package \emph{NumPy}\cite{scipy} durchgeführt. Fehlerrechnungen werden mit dem für \emph{Python} geschriebenen package \emph{Uncertainties}\cite{uncertainties} ausgeführt.\\

Die Nullmessung ergibt $N_\textrm{\Delta t}=460\pm22$ Detektionen in einem Zeitraum von $\Delta t = 900s$.
Durch die Annahme eines linearen Zusammenhangs der Nullmessung werden die Messwerte für Indium und Silber um den Wert der Nullmessung korrigiert. Ferner werden im Folgenden sämtliche Messwerte mit dem Fehler gemäß \eqref{eq:fehlerN} ausgewertet.
% Bei einem Zeitintervall $\Delta t_\textrm{Ind} = \SI{220}{\seconds}$ bei der Messung von Indium, ergeben sich ein Fehler von 5


\subsection{Indium}
\label{sec:Indium}
In Abbildung \ref{fig:Indium_plot} sind die aufgenommenen Messdaten aus Tabelle $\ref{table:Indium}$ in einem halblogarithmischen Diagramm aufgetragen.
\begin{figure}
  \centering
  \includegraphics{build/Indium_plot.pdf}
  \caption{ Messwerte und Regressionsgerade für Indium.}
  \label{fig:Indium_plot}
\end{figure}

\input{build/Tabelle_Indium_texformat.tex}

Die eingezeichnete lineare Ausgleichsrechnung liefert für die Steigung $m$ und den y-Achsendurchgang $b$ die Ergebnisse
\begin{align*}
  m_\textrm{\ce{^{116}_{}In}} &=  \input{build/Indium_m.tex} \\
  b_\textrm{\ce{^{116}_{}In}} &=  \input{build/Indium_b.tex} \;.
\end{align*}

Im Vergleich mit Gleichung $\ref{eq:Nvont}$ entspricht $\lambda = -m_\textrm{\ce{^{116}_{}In}}$. %und $A = b_\textrm{\ce{^{116}_{}In}}$.
Daraus ergibt sich die Halbwertszeit $T_\textrm{\ce{^{116}_{}In}}$ für Indium nach Gleichung $\ref{eq:Halbwertszeit}$ zu
\begin{align*}
  T_\textrm{\ce{^{116}_{}In}} &= \input{build/Halbwertszeit_Indium_h.tex} \;.
\end{align*}
Weiter ergibt sich aus Gleichung $\ref{eq:Nvont}$ die Relation
\begin{align}
  N_0 &= \exp{b} \;,
  \label{eq:expb}
\end{align}
da $b$ aus einem halblogarithmischen Diagramm entnommen wurde. Nach der obigen Gleichung folgt
\begin{align*}
  N_0 &= \input{build/Startwert_Indium.tex} \;.
\end{align*}
Es ist zu beachten das $N_0$ hier und im folgenden nur ein Teil der Gesamtstrahlungsmenge ist, da das Geiger-Müller-Zählrohr nur einen räumlich festen Anteil Strahlung detektiert.

\subsection{Silber}
\label{sec:Silber}
Da das verwendete Silber aus zwei Isotopen besteht, müssen die aufgenommenen Daten in Abbildung $\ref{fig:Silber_plot}$ in zwei Abbildungen aufgeteilt werden, sodass der langsame und der schnelle Zerfall getrennt berechnet werden können. Die dazu verwendeten Messdaten sind in Tabelle \ref{table:Silber} aufgelistet.

\input{build/Tabelle_Silber_texformat.tex}

\begin{figure}[H]
  \centering
  \includegraphics{build/Silber_plot.pdf}
  \caption{Messdaten und Fitergebnis.}
  \label{fig:Silber_plot}
\end{figure}

Für die Trennung der beiden Zerfälle wird der Zeitpunkt $t^* = \SI{153}{\second}$ gewählt, da sich die gedachten Geraden der beiden Zerfälle an diesem Punkt schneiden.
Somit ist der Zerfall von $\ce{^{108}_{}Ag}$ im Bereich von $t > t^*$. In dem Bereich von $t < t^*$ jedoch sind die beiden Zerfälle addiert.
Aus diesem Grund werden im Folgenden zuerst durch eine Ausgleichsrechnung die Parameter des langsamen Zerfalls bestimmt, sodass im nächten Schritt diese aus den Messwerten im Bereich $t < t^*$ rausgerechnet werden können.

% Somit sind die Messwerte für $\ce{^{110}_{}Ag}$ in  Abbildung $\ref{fig:Silber110}$  und die Messwerte für  $\ce{^{108}_{}Ag}$ in Abbildung $\ref{fig:Silber110}$  dargestellt. Zusätzlich ist jeweils die Regressionsgerade in die entsprechende Abbildung eingezeichnet.
In der folgenden Abbildung ist der Zerfall von $\ce{^{108}_{}Ag}$ in einem halblogarithmischen Diagramm dargestellt.
\begin{figure}[H]
  \centering
  \includegraphics{build/Silber_plot_108.pdf}
  \caption{Messdaten und Fitergebnis von $\ce{^{108}_{}Ag}$.}
  \label{fig:Silber108}
\end{figure}
Wie in Kapitel $\ref{sec:Indium}$ beschrieben wird aus den Parametern der Regressionsgerade die Halbwertszeit und der Startwert $N_0$ bestimmt
\begin{align*}
  m_\textrm{\ce{^{108}_{}Ag}} &= \input{build/Silber_108_m.tex}\\
  b_\textrm{\ce{^{108}_{}Ag}} &= \input{build/Silber_108_b.tex} \\
  T_\textrm{\ce{^{108}_{}Ag}} &= \input{build/Halbwertszeit_Silber_108.tex}\\
  N_0 &= \input{build/Startwert_Silber_108.tex} \;.
\end{align*}

In Abbildung $\ref{fig:Silber110}$ ist der von $\ce{^{110}_{}Ag}$ dargestellt. Hierzu wurde der langsamere Zerfall von Abbildung $\ref{fig:Silber108}$ vom Gesamtzerfall abgezogen gemäß Gleichung $\ref{eq:MARIUS}$.

\begin{figure}[H]
  \centering
  \includegraphics{build/Silber_plot_110.pdf}
  \caption{Messdaten und Fitergebnis von $\ce{^{110}_{}Ag}$.}
  \label{fig:Silber110}
\end{figure}
Aus der Regressionsgeraden ergeben sich die Werte
\begin{align*}
  m_\textrm{\ce{^{110}_{}Ag}} &= \input{build/Silber_110_m.tex}\\
  b_\textrm{\ce{^{110}_{}Ag}} &= \input{build/Silber_110_b.tex} \;.
\end{align*}
Mit dem in Kapitel $\ref{sec:Indium}$ beschriebenen Verfahren wird die Halbwertszeit $T_\textrm{110}$ sowie der Startwert der Teilchenzahl $N_0$ für $\ce{^{110}_{}Ag}$ nach Gleichung $\ref{eq:Halbwertszeit}$ und $\ref{eq:expb}$ berechnet. Daraus folgt
\begin{align*}
  T_\textrm{\ce{^{110}_{}Ag}} &= \input{build/Halbwertszeit_Silber_110.tex}\\
  N_0 &= \input{build/Startwert_Silber_110.tex} \;.
\end{align*}


% \begin{figure}[H]
%   \centering
%   \includegraphics{build/Silber_plot_108.pdf}
%   \caption{Messdaten und Fitergebnis von $\ce{^{108}_{}Ag}$.}
%   \label{fig:Silber108}
% \end{figure}
% Wie zuvor beschrieben wird aus den Parametern der Regressionsgerade die Halbwertszeit und der Startwert $N_0$ bestimmt
% \begin{align*}
%   m_\textrm{\ce{^{108}_{}Ag}} &= \input{build/Silber_108_m.tex}\\
%   b_\textrm{\ce{^{108}_{}Ag}} &= \input{build/Silber_108_b.tex} \\
%   T_\textrm{\ce{^{108}_{}Ag}} &= \input{build/Halbwertszeit_Silber_108.tex}\\
%   N_0 = \input{build/Startwert_Silber_108.tex} \;.
% \end{align*}


Schlussendlich ist in Abbildung \ref{fig:Silber108_final} die Summe der beiden Ausgleichsrechnung

\begin{align}
  f(t) &= N_\textrm{\ce{^{108}_{}Ag}} \cdot \exp{(-\lambda_\textrm{\ce{^{108}_{}Ag} \cdot t)} + N_\textrm{\ce{^{110}_{}Ag}}} \cdot \exp{(-\lambda_\textrm{\ce{^{110}_{}Ag}} \cdot t)} \\
       &= \input{build/Startwert_Silber_108_nom.tex} \cdot \exp{- \input{build/lambda_Si_108.tex} \cdot t} + \input{build/Startwert_Silber_110_nom.tex} \cdot \exp{- \input{build/lambda_Si_110.tex} \cdot t}
\end{align}

abgebildet.
\begin{figure}[H]
  \centering
  \includegraphics{build/Silber_mit_Ausgleichsgrade.pdf}
  \caption{Messdaten und Fitergebnis von $\ce{^{108}_{}Ag}$ und $\ce{^{110}_{}Ag}$.}
  \label{fig:Silber108_final}
\end{figure}

% % Examples
% \begin{equation}
%   U(t) = a \sin(b t + c) + d
% \end{equation}
%
% \begin{align}
%   a &= \input{build/a.tex} \\
%   b &= \input{build/b.tex} \\
%   c &= \input{build/c.tex} \\
%   d &= \input{build/d.tex} .
% \end{align}
% Die Messdaten und das Ergebnis des Fits sind in Abbildung~\ref{fig:plot} geplottet.
%
% %Tabelle mit Messdaten
% \begin{table}
%   \centering
%   \caption{Messdaten.}
%   \label{tab:data}
%   \sisetup{parse-numbers=false}
%   \begin{tabular}{
% % format 1.3 bedeutet eine Stelle vorm Komma, 3 danach
%     S[table-format=1.3]
%     S[table-format=-1.2]
%     @{${}\pm{}$}
%     S[table-format=1.2]
%     @{\hspace*{3em}\hspace*{\tabcolsep}}
%     S[table-format=1.3]
%     S[table-format=-1.2]
%     @{${}\pm{}$}
%     S[table-format=1.2]
%   }
%     \toprule
%     {$t \:/\: \si{\milli\second}$} & \multicolumn{2}{c}{$U \:/\: \si{\kilo\volt}$\hspace*{3em}} &
%     {$t \:/\: \si{\milli\second}$} & \multicolumn{2}{c}{$U \:/\: \si{\kilo\volt}$} \\
%     \midrule
%     1.7 & 10 \\
2.3 & 20 \\
3.5 & 30 \\
4.4 & 40 \\

%     \bottomrule
%   \end{tabular}
% \end{table}
%
% % Standard Plot
% \begin{figure}
%   \centering
%   \includegraphics{build/plot.pdf}
%   \caption{Messdaten und Fitergebnis.}
%   \label{fig:plot}
% \end{figure}
%
% 2x2 Plot
% \begin{figure*}
%     \centering
%     \begin{subfigure}[b]{0.475\textwidth}
%         \centering
%         \includegraphics[width=\textwidth]{Abbildungen/Schaltung1.pdf}
%         \caption[]%
%         {{\small Schaltung 1.}}
%         \label{fig:Schaltung1}
%     \end{subfigure}
%     \hfill
%     \begin{subfigure}[b]{0.475\textwidth}
%         \centering
%         \includegraphics[width=\textwidth]{Abbildungen/Schaltung2.pdf}
%         \caption[]%
%         {{\small Schaltung 2.}}
%         \label{fig:Schaltung2}
%     \end{subfigure}
%     \vskip\baselineskip
%     \begin{subfigure}[b]{0.475\textwidth}
%         \centering
%         \includegraphics[width=\textwidth]{Abbildungen/Schaltung4.pdf}    % Zahlen vertauscht ... -.-
%         \caption[]%
%         {{\small Schaltung 3.}}
%         \label{fig:Schaltung3}
%     \end{subfigure}
%     \quad
%     \begin{subfigure}[b]{0.475\textwidth}
%         \centering
%         \includegraphics[width=\textwidth]{Abbildungen/Schaltung3.pdf}
%         \caption[]%
%         {{\small Schaltung 4.}}
%         \label{fig:Schaltung4}
%     \end{subfigure}
%     \caption[]
%     {Ersatzschaltbilder der verschiedenen Teilaufgaben.}
%     \label{fig:Schaltungen}
% \end{figure*}
