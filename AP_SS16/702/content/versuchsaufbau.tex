\section{Versuchsaufbau und Durchführung}
\label{sec:Versuchaufbau}

Die für die Aktivierung der Proben nötigen Neutronen kommen in der Natur nicht vor, weil sie instabil sind. Dies folgt aus dem Bestreben der Natur, Zustände minimaler Energien einzustellen. Neutronen zerfallen deshalb in Protonen, deren Masse geringer ist, vgl. \eqref{eq:betazerfall}. Die Neutronen werden daher eigens für den Versuch erzeugt, indem Be-Kerne mit $\alpha$-Teilchen, welche wiederum aus dem Zerfall von $\ce{^{226} Ra}$ stammen, beschossen werden. Dies geschieht in einem Kessel, dessen Innenwand mit Paraffin beschichtet ist. Die durch Stöße mit dem Paraffin abgebremsten Neutronen gelangen zu den ebenfalls in den Kessel eingelassenen Probekörpern. Für den Versuch wird Indium und Silber als Ausgangsmaterial verwendet. In der eigentlichen Messapparatur wird ein geometrisch konstanter Bruchteil der Zerfälle mit einem Geiger-Müller-Zählrohr detektiert. Probe und Zählrohr sind dabei von einem Bleiquader umgeben. Dieser dient dem Schutz der Apparatur vor störender natürlicher Strahlung.
Ein Zählwerk mit zwei Anzeigevorrichtungen registriert die Signale des Zählrohrs pro Zeiteinheit $\Delta t$. Diese Zeiteinheit lässt sich durch einen Zeitgeber mit einer relativen Genauigkeit von $10^{-5}$ einstellen.

Da die natürliche Strahlung nicht gänzlich unterbunden werden kann, wird zunächst dieser sogenannte Nulleffekt gemessen, indem die Messzeit ohne das Befüllen mit einer Probe auf $\SI{900}{\second}$ eingestellt und der resultierende Messwert notiert wird. Die Messzeit wird hierbei hoch gewählt, um den statistischen Fehler klein zu halten. Nun wird eine der jeweiligen Probe angepasste Messzeit $\Delta t$ eingestellt, bevor die aktivierten Proben zügig zu der  Messvorrichtung transportiert werden. Die Messung wird gestartet und es werden die folgenden Messdauern gewählt.
\begin{align*}
  \text{Indium} & & \Delta t &= 220\si{\second} & & t_\text{ges} \approx 60\si{\minute} \\
  \text{Silber} & & \Delta t &= 9\si{\second} & & t_\text{ges} \approx 7\si{\minute}
\end{align*}
