\section{Diskussion}
\label{sec:Diskussion}
	Die starke Abweichung des Prismeninnenwinkels ist der schlechten Justierbarkeit des Prismas geschuldet: Ein paralleler Lichteinfall ließ sich nicht sicherstellen, da die Spitze des Prisma nach Augenmaß in den Lichtstrahl rotiert werden musste. 

	Eine Diskussion der restlichen Messdaten erübrigt sich, da diese nicht von den Autoren stammen. Die Messdaten zu den Winkel scheinen fehlerbehaftet zu sein, was sich in der unrealistischen Abbeschen Zahl und der recht hohen Steigung der Dispersionskurve zeigt.

	Es liegen keine Messdaten zu $\eta$ von den Autoren vor, da in diesem Schritt des Versuches am Versuchsaufbau keine validen Winkel nach der Anleitung des Praktikumsassistenten gemessen werden konnten, weder von den Autoren noch vom Assistenten.



\newpage
\section{Literaturangabe}
\label{sec:Literatur}

Bilder und Daten aus dem Skript zu \emph{Prismadispersion}, Versuch 402, TU Dortmund, abrufbar auf:\\
\url{http://129.217.224.2/HOMEPAGE/PHYSIKER/BACHELOR/AP/SKRIPT/V402.pdf}\\(Stand 19.04.16)\par

NIST-Daten zu den Spektrallinien\\
\url{http://physics.nist.gov/PhysRefData/Handbook/Tables/cadmiumtable2_a.htm}\\
\url{http://physics.nist.gov/PhysRefData/Handbook/Tables/mercurytable2.htm}\\(Stand 19.04.16)

Schott-Daten zu Flintglas aus Peter Hartmann, Ralf Jedamzik, Steffen Reichel und Bianca Schreder, \emph{Optical glass and glass ceramic historical aspects and recent developments: a Schott view} Appl. Opt. 49, D157-D176 (2010).