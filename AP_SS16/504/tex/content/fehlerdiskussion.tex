\section{Diskussion}
\label{sec:Diskussion}

Die ermittelten Kennlinien von Diode~2 entsprechen zwar dem erwarteten Kurvenverlauf, sind aber aufgrund des sehr schnell\footnote{Bei der letzten Messung wurde der Sättigungsstrom schon nach vier Messwerten erreicht.} erreichten Sättigungsstroms nicht besonders gut geeignet, um den Langmuir-Schottkyschen Exponenten zu bestimmen. Trotzdem weißt der berechnete Wert \emph{nur} einen Fehler von $9.6\%$ auf. Ein möglicher Grund dafür ist der maximal erlaubte Heizstrom der Diode von $\SI{2.0}{\ampere}$. Mit einem höheren Heizstrom hätte eventuell eine höhere Genauigkeit erreicht werden können.

Die Kathodentemperatur, die mit Hilfe des Anlaufstromgebietes ermittelt wurde, führt zu einem realistischen Wert mit $T = (2454 \pm 21) \si{\kelvin}$ bei $\SI{2.5}{\ampere}$ Heizstrom. Dabei ist zu bemerken, dass die Messung bei Diode~2 nicht möglich war\footnote{Die Werte schwankten extrem und machten ein ablesen unmöglich. Schon ein kleiner Lufthauch veränderte die Werte stark etc. .} und die Autoren deshalb ein anderes Gerät (Diode~1) zur Messung verwendeten.

Die Kathodentemperaturbestimmung aus den Kennlinien liefert im Vergleich zu der vorherigen Messung niedrige Werte (die größte Differenz beträgt ca. $834\si{\kelvin}$). Die Abweichung ist durch den im Vergleich zu Diode~1 niedrigeren maximal Heizstrom verursacht und liegt im Erwartungsbereich der Autoren.

Umso verwunderlicher ist die extreme Abweichung der Austrittsarbeit von Wolfram verglichen mit dem Theoriewert. Sie beträgt ca. $63\%$ und lässt eigentlich auf einen Fehler bei der Messung bzw. Durchführung schließen oder einen Defekt am Gerät, obwohl die anderen Werte eher gegen einen solchen sprechen.

Insgesamt scheint Diode~2 für diesen Versuch eher ungeeignet zu sein.




\newpage
\section{Literaturangabe}
\label{sec:Literatur}

Bilder und Daten aus dem Skript zu \emph{Thermische Elektronenemission}, Versuch 504, TU Dortmund, abrufbar auf:\\
\url{http://129.217.224.2/HOMEPAGE/PHYSIKER/BACHELOR/AP/SKRIPT/V504.pdf}\\(Stand 16.05.16)\par

Bild zur Fermiverteilung von Wikipedia/Fermi–Dirac statistics, abgerufen am 16.05.16 unter \url{https://en.wikipedia.org/wiki/Fermi%E2%80%93Dirac_statistics}\par

Bild zur Strom-Spannungskurve aus Gerthsen Physik 24. Auflage Meschede, Springer 2010\par


\subsection{Theoriewerte der physikalischen Konstanten}
\label{sub:theoriewerte_der_physikalischen_konstanten}


NIST, \emph{Elementarladung}, abrufbar auf:\\
\url{http://physics.nist.gov/cgi-bin/cuu/Value?e}\\
(Stand 16.05.16)\par

NIST, \emph{Elektronen Masse}, abrufbar auf:\\
\url{http://physics.nist.gov/cgi-bin/cuu/Value?me}
(Stand 16.05.16)\par

NIST, \emph{Plancksches Wirkungsquantum}, abrufbar auf:\\
\url{http://physics.nist.gov/cgi-bin/cuu/Value?h|search_for=planck}
(Stand 16.05.16)\par

NIST, \emph{Boltzmannkonstante}, abrufbar auf:\\
\url{http://physics.nist.gov/cgi-bin/cuu/Value?k|search_for=boltzmann}
(Stand 16.05.16)\par
