\newpage
\section{Aufbau und Durchführung}\label{sec:aufbau-und-durchfuehrung}

In diesem Versuch werden folgende Zusammenhänge gemessen:
\begin{enumerate}
	\item Es werden zu verschiedenen Heizleistungen die jeweiligen Kennlinien der Röhre bestimmt, indem bei jeder Heizleistung etwa 50 Strom-Spannungs-Paare gemessen werden (Anodenstrom und Saugspannung). Daraus lässt sich der Sättigungsstrom bestimmen und der Gültigkeitsbereich der Raumladungsformel abschätzen. Zudem wird aus der Heizleistung die Kathodentemperatur abgeschätzt.

	\item Bei größtmöglicher Heizleistung wird ein Gegenfeld angelegt und der Anlaufstrom bestimmt (durch Messung des Anodenstroms und des Gegenfeldes). Daraus lässt sich die Kathodentemperatur bestimmen.

	\item Schließlich wird aus den berechneten Kathodentemperatur-Anodenstrom-Paaren die Austrittsarbeit des Kathodenmaterials berechnet.
\end{enumerate}

Der Aufbau ist in Abbildung~\ref{fig:_figures_diode_aufbau_png} zu sehen. Die Saugspannung kann variiert werden und an der Spannungsquelle abgelesen werden, das angeschlossene Messgerät misst den Strom zwischen Anode und Kathode.