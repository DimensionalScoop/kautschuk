\subsection{Durchführung}
\label{sec:durchführung}
\subsection{Untersuchung des Acrylblocks}
Zunächst werden die Ausmaße des Acrylblocks, sowie die Tiefe der Bohrungen, über eine Schieblehre bestimmt und notiert.
Daraufhin wird die Oberseite des Blocks mit einer Ultraschallsonde abgefahren, während mittels Impuls-Echo-Verfahren und A-Scan die Laufzeit des Schalls begrenzt durch die Bohrungen (bzw. Fehlstellen) bestimmt wird.
Als Koppelmittel wird hier wie in den folgenden Scans bidestilliertes Wasser verwendet.
Nach dem Scan wird der Acrylblock umgedreht und es wird erneut gescannt.
Dieses Verfahren wird nun für einen B-Scan wiederholt, wobei darauf geachtet wird, dass die Sonde mit möglichst konstanter Geschwindigkeit über den Block gefahren wird.

\subsection{Untersuchung der Herzfrequenz am Herzmodell}
Zuerst wird der Durchmesser der Kreismembran des Herzmodells ausgemessen.
Dann wird das Modell zu einem Drittel mit Wasser gefüllt.
Die Sonde wird so angebracht, dass sie nur die Oberfläche des Wassers berührt.
Sind diese Vorbereitungen getroffen, wird ein TM-Scan gestartet.
Während des Scans wird hydraulisch per Hand möglichst periodisch und impulsartig Luft in die Membran eingeschleust, welches einen Herzschlag simulieren soll.
