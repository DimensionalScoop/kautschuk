\section{Diskussion}
\label{sec:Diskussion}
Bei der Bestimmung des Auflösungsvermögens von $\SI{0.1}{\milli\metre}$ konnten die beiden kleinsten Fehlstellen auseinander gehalten werden.
Um eine Verbesserung der Auflösung zu ermöglichen, müsste statt der verwendeten $\SI{1}{\mega\hertz}$-Sonde eine Sonde mit höherer Frequenz verwendet werden.\\
Bei dem Vergleich der bestimmten Durchmesser der Fehlstellen ist auffällig, dass die mit dem B-Scan bestimmten Werte im Vergleich zu den mit den A-Scan bestimmten Werte durchgängig größer sind.
Zudem können repräsentativ die mit der Schieblehre bestimmten oberen Abstände mit den A- bzw. B-Scan Werten verglichen werden.
Hier zeigt sich, dass die Werte aus dem A-Scan eine Abweichung von durchschnittlich
\begin{align*}
  \increment D_\text{A} &= \input{build/D_o_rel_a.tex},
\end{align*}
die aus dem B-Scan eine Abweichung von
\begin{align*}
  \increment D_\text{B} &= \input{build/D_o_rel_b.tex},
\end{align*}
besitzen.
Eine weitere Auffälligkeit zeigt sich bei der Betrachtung der berechneten Höhe des Quaders.
Mittels Schieblehre wurde eine Höhe von
\begin{align*}
  h_\text{gem} &= \SI{8.035}{\centi\metre}
\end{align*}
ermittelt, der A-Scan ergab eine Höhe von
\begin{align*}
  h_\text{A} &= \input{hoehe_mess.tex},
\end{align*}
der B-Scan eine Höhe von\begin{align*}
  h_\text{B} &= \input{hoehe_mess2.tex}.
\end{align*}
Dies entspricht einer Abweichung respektive
\begin{align*}
  \increment h_\text{A} &= \input{hoehe_mess_rel.tex}
\end{align*}
bzw.
\begin{align*}
  \increment h_\text{B} &= \input{hoehe_mess_rel2.tex}.
\end{align*}
Aufgrund der Beobachtung, dass bereits diese Messung einen systematischen Fehler ausweist, kann davon ausgegangen werden, dass die Eintrittslaufzeit und somit die Schutzschicht nicht korrekt berücksichtigt wurden.
Dies liegt an der Tatsache, dass sowohl im A- als auch im B-Scan dieser Wert nicht eindeutig abgelesen werden konnte.
Insgesamt erscheint die Messung der Fehlstellen mit dem A-Scan als zuverlässiger, was sich in den geringeren Abweichungen widerspiegelt.\\
Im Vergleich mit Literaturwerten scheint das bestimmte Herzzeitvolumen ein realistischer Wert zu sein.
