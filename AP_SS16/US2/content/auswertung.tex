\section{Auswertung}
\label{sec:Auswertung}

\subsection{Ausmessung des Acryl-Blocks mittels A-Scan}

Die Ausmessung des Blocks via Schieblehre und A-Scan führt zu den Daten, welche in Tabelle \ref{table:1} dargestellt sind.
\input{build/Tabelle_a_texformat.tex}
Hierbei beschreibt $D_\text{oben}$ den mit der Schieblehre bestimmte Abstand von der oberen Kante der Löcher bis zur Kante des Acrylblocks.\\
Zur Bestimmung der Durchmesser mittels A-Scan wird zunächst aus der Eintrittslaufzeit
\begin{align*}
  t_\text{start} &= \input{build/t_start.tex}
\end{align*}
und aus der Austrittslaufzeit
\begin{align*}
  t_\text{end} &= \input{build/t_end.tex}
\end{align*}
die Höhe des Blockes zu
\begin{align*}
  h &= \input{build/hoehe_mess.tex}
\end{align*}
bestimmt.
Aus den ermittelten Laufzeiten werden, unter Berücksichtigung der Eintrittslaufzeit, die Abstände $D_\text{oben,gem}$ der oberen Enden der Fehlstellen zu dem oberen Ende des Blockes sowie die Abstände $D_\text{unten,gem}$ der unteren Enden der Fehlstellen zu dem unteren Ende des Blockes bestimmt.
Für diese Berechnung wird eine Phasengeschwindigkeit in Acryl von $\SI{2730}{\metre\per\second}$ im Weg-Zeit-Gesetz verwendet. \cite{schall}\\
Zusammen mit der ermittelten Höhe des Blockes ergeben sich die Durchmesser der Fehlstellen $D_\text{loch,gem}$.\\
Die Messung der kleinen Fehlstellen $10$ und $11$ ergibt einen Abstand von $\Delta D = \SI{1.7}{\milli\metre}$.
Zudem ist anzumerken, dass der tatsächliche Abstand der beiden letzten Fehlstellen $\Delta D = \SI{1.8}{\milli\metre}$ entspricht.
Folglich besitzt der A-Scan eine Auflösung von etwa $\SI{0.1}{\milli\metre}$.\\

\subsection{Ausmessung des Acryl-Blocks mittels B-Scan}
Das Ergebnis des B-Scans von oben ist in Abbildung \ref{figure:1} wiedergegeben; das des unteren in Abbildung \ref{figure:2}.

\begin{figure}[H]
  \centering
  \includegraphics[height=7.5cm]{messdaten/b_oben.png}
  \caption{Ergebnis des B-Scans von oben.}
  \label{figure:1}
\end{figure}

\begin{figure}[H]
  \centering
  \includegraphics[height=7.5cm]{messdaten/b_unten.png}
  \caption{Ergebnis des B-Scans von unten.}
  \label{figure:2}
\end{figure}

Aus diesen beiden Grafiken ergeben sich die Messdaten aus Tabelle \ref{table:2}.
\input{build/Tabelle_b_texformat.tex}
Es wird wiederum der mit der Schieblehre bestimmte Wert $D_\text{oben}$ angegeben, außerdem die aus der Laufzeit im B-Scan bestimmten Abstände der Löcher nach oben, nach unten und der daraus resultierende Fehlstellendurchmesser.
Hierbei wird die Phasengeschwindigkeit in Acryl sowie die aus der Eintrittszeit
\begin{align*}
  t_\text{start} &= \input{build/t_start2.tex}
\end{align*}
und der Austrittszeit
\begin{align*}
  t_\text{end} &= \input{build/t_end2.tex}
\end{align*}
bestimmte Blockhöhe von
\begin{align*}
  h &= \input{build/hoehe_mess2.tex}
\end{align*}
berücksichtigt.
In der Diskussion werden die Ergebnisse aus A- und B-Scan verglichen.

\subsection{Untersuchung des Herzmodells}

\subsubsection{Bestimmung der Laufzeit eines Echos in Wasser}

Zunächst wird ein A-Scan, der in Abbildung \ref{figure:5} dargestellt ist, durchgeführt.

\begin{figure}[H]
  \centering
  \includegraphics[height=9cm]{messdaten/herz_a.png}
  \caption{A-Scan der Herzmembran.}
  \label{figure:5}
\end{figure}
Es werden die Laufzeiten
\begin{align*}
  t_1 &= \input{build/s_1.tex}.
\end{align*}
sowie
\begin{align*}
  t_2 &= \input{build/s_2.tex}.
\end{align*}
abgelesen, woraus sich ein gemessener Abstand von
\begin{align*}
  s &= \input{build/h_bla.tex}.
\end{align*}
ergibt.
Bei dieser Berechnung wird die Phasengeschwindigkeit in Wasser
\begin{align*}
  c_{\text{Wasser}} &= \input{build/c_wasser.tex}
\end{align*}
berücksichtigt.\cite{schall}

\subsubsection{Bestimmung der Herzfrequenz und des Herzzeitvolumens}
Ausgangspunkt der Messung der Herzfrequenz und des Herzvolumens ist der TM-Scan aus Abbildung \ref{figure:3}.

\begin{figure}[H]
  \centering
  \includegraphics[height=9cm]{messdaten/herz.png}
  \caption{TM-Scan des schlagenden Herzmodells.}
  \label{figure:3}
\end{figure}

Dabei ergeben sich für die Amplituden die Werte
\begin{align*}
  A_1 &= \SI{75}{\micro\second},\\
  A_2 &= \SI{77}{\micro\second},\\
  A_3 &= \SI{77}{\micro\second},\\
  A_4 &= \SI{77}{\micro\second},\\
  A_5 &= \SI{77}{\micro\second}.
\end{align*}
Die zeitlichen Abstände der Herzschläge betragen
\begin{align*}
  t_1 &= \SI{2,10}{\second},\\
  t_2 &= \SI{2,04}{\second},\\
  t_3 &= \SI{2,04}{\second},\\
  t_4 &= \SI{2,06}{\second}.
\end{align*}
Daraus ergibt sich eine mittlere Herzfrequenz von
\begin{align*}
  f_{\text{Herz}} &= \input{build/hf.tex}.
\end{align*}
Der endsystolische Durchmesser (ESD) wird über die Formel
\begin{equation}
  ESD = \frac{1}{2} c_{\text{Wasser}} \cdot \text{A}
\end{equation}
gemittelt auf
\begin{align*}
  ESD &= \input{build/ESD.tex}
\end{align*}
bestimmt.
Damit lässt sich nun das Herzvolumen, welches auf eine Kugel mittels
\begin{equation}
  V_{\text{Herz}} = \frac{4\pi}{3} \left(\frac{ESD}{2} \right)³
\end{equation}
genähert wird, zu
\begin{align*}
  V_{\text{Herz}} &= \input{build/ESV.tex}
\end{align*}
berechnet.
Daraus folgt ein Herzzeitvolumen,
\begin{equation}
  HZV = V_{\text{Herz}} \cdot f_{\text{Herz}},
\end{equation}
von etwa
\begin{align*}
  HZV &= \input{build/HZV.tex},
\end{align*}
wenn das enddiastolische Volumen für diesen Modellfall als $0$ angenommen wird.

% % Examples
% \begin{equation}
%   U(t) = a \sin(b t + c) + d
% \end{equation}
%
% \begin{align}
%   a &= \input{build/a.tex} \\
%   b &= \input{build/b.tex} \\
%   c &= \input{build/c.tex} \\
%   d &= \input{build/d.tex} .
% \end{align}
% Die Messdaten und das Ergebnis des Fits sind in Abbildung~\ref{fig:plot} geplottet.
%
% %Tabelle mit Messdaten
% \begin{table}
%   \centering
%   \caption{Messdaten.}
%   \label{tab:data}
%   \sisetup{parse-numbers=false}
%   \begin{tabular}{
% % format 1.3 bedeutet eine Stelle vorm Komma, 3 danach
%     S[table-format=1.3]
%     S[table-format=-1.2]
%     @{${}\pm{}$}
%     S[table-format=1.2]
%     @{\hspace*{3em}\hspace*{\tabcolsep}}
%     S[table-format=1.3]
%     S[table-format=-1.2]
%     @{${}\pm{}$}
%     S[table-format=1.2]
%   }
%     \toprule
%     {$t \:/\: \si{\milli\second}$} & \multicolumn{2}{c}{$U \:/\: \si{\kilo\volt}$\hspace*{3em}} &
%     {$t \:/\: \si{\milli\second}$} & \multicolumn{2}{c}{$U \:/\: \si{\kilo\volt}$} \\
%     \midrule
%     1.7 & 10 \\
2.3 & 20 \\
3.5 & 30 \\
4.4 & 40 \\

%     \bottomrule
%   \end{tabular}
% \end{table}
%
% % Standard Plot
% \begin{figure}
%   \centering
%   \includegraphics{build/plot.pdf}
%   \caption{Messdaten und Fitergebnis.}
%   \label{fig:plot}
% \end{figure}
%
% 2x2 Plot
% \begin{figure*}
%     \centering
%     \begin{subfigure}[b]{0.475\textwidth}
%         \centering
%         \includegraphics[width=\textwidth]{Abbildungen/Schaltung1.pdf}
%         \caption[]%
%         {{\small Schaltung 1.}}
%         \label{fig:Schaltung1}
%     \end{subfigure}
%     \hfill
%     \begin{subfigure}[b]{0.475\textwidth}
%         \centering
%         \includegraphics[width=\textwidth]{Abbildungen/Schaltung2.pdf}
%         \caption[]%
%         {{\small Schaltung 2.}}
%         \label{fig:Schaltung2}
%     \end{subfigure}
%     \vskip\baselineskip
%     \begin{subfigure}[b]{0.475\textwidth}
%         \centering
%         \includegraphics[width=\textwidth]{Abbildungen/Schaltung4.pdf}    % Zahlen vertauscht ... -.-
%         \caption[]%
%         {{\small Schaltung 3.}}
%         \label{fig:Schaltung3}
%     \end{subfigure}
%     \quad
%     \begin{subfigure}[b]{0.475\textwidth}
%         \centering
%         \includegraphics[width=\textwidth]{Abbildungen/Schaltung3.pdf}
%         \caption[]%
%         {{\small Schaltung 4.}}
%         \label{fig:Schaltung4}
%     \end{subfigure}
%     \caption[]
%     {Ersatzschaltbilder der verschiedenen Teilaufgaben.}
%     \label{fig:Schaltungen}
% \end{figure*}
