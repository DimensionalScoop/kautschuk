\newpage

\section{Gaußsche Fehlerrechnung}\label{gausssche-fehlerrechnung}

\subsection{Berechnung der
Standardabweichung}\label{berechnung-der-standardabweichung}

Alle Messwerte sind als empirische Mittelwerte mit ihrer geschätzten
Standardabweichung des Mittelwertes angegeben. Diese unterschätzt die
wahre Standardabweichung, da die Wurzel aus der geschätzten Varianz
gezogen wird. Der arithmetische Mittelwert ist definiert als

\begin{align}
    \bar x = \frac{1}{n} \  \sum_{i=1}^n x_i \  .
    \label{equ:mean}
\end{align}

Die geschätzte Standardabweichung ist gegeben durch

\begin{align}
    s = \sqrt{\frac{1}{n-1} \sum_{i=1}^n (x_i - \bar x)^2}
    \label{equ:s}
\end{align}

mit der geschätzten Standardabweichung des Mittelwertes als

\begin{align}
    \Delta \bar x = \frac{s}{ \sqrt{n} }
    \label{equ:stdofmean}
\end{align}

\subsection{Gaußsche
Fehlerfortpflanzung}\label{gausssche-fehlerfortpflanzung}

Das Berechnen von Funktionen mit fehlerbehafteten Parametern erfolgt
mittels der Gauß'schen Fehlerfortpflanzung

\begin{align}
    \Delta f = \sqrt{ \sum_{i=1}^n \left( \frac{\diff f}{\diff y_i} \right)^2 (\Delta y_i)^2} \qquad \qquad \mathrm{mit} \quad f(y_1, \ldots , y_n)
    \label{equ:gauss}
\end{align}
