\section{Diskussion}
\label{sec:Diskussion}


Die starke Varianz der Fehler unter den Ladungen ist darauf zurückzuführen, dass für manche Tröpfchen nur eine Messung vorgenommen wurde. Während für andere bis zu fünf Messungen des selben Tröpfchens möglich waren, die dann gemittelt werden konnten. Generell war es schwierig ausreichend geladene Teilchen zu finden, die außerdem über den gesamten Zeitraum der Messung stabil blieben. Die Messung der Gleichgewichtsgeschwindigkeit $v_0$ war nur für sechs von 28 Tröpfchen möglich.
Die Tröpfchen, für die eine Messung von $v_0$ möglich war, mussten auch mit der Annahme einer Messungenauigkeit von $25\%$ aufgrund von Gleichung~\eqref{equ:test} aussortiert werden.
Aufgrund der fehlenden $v_0$ Messungen ist es daher den Autoren nicht möglich, weitere eventuell fehlerhafte Messwerte (d.h. Öltröpfchen, die während der Messung ihre Ladung änderten) auszusortieren.

Wie in Abbildung~\ref{fig:ladung} an den magentafarbenen Messwerten zu sehen ist, ist zwar eine ``Gruppierung'' der gemittelten Ladungen um bestimmte Werte zu erkennen, aber sie ist nicht eindeutig und macht die Zuordnung einiger Werte zu einer bestimmten ``Gruppe'' schwierig. Daher wurde ein an den \emph{Euklidischen-Algorithmus} angelehnter Algorithmus für die Ermittlung des größten gemeinsamen Teilers programmiert. Die Abweichung der berechneten Elementarladung zum Theoriewert um ca. $8.8\%$ ist daher akzeptabel.

Aufgrund der Antiproportionalität zwischen der Elementarladung und der Avogadrokonstante ist es nicht verwunderlich, dass auch hier die Abweichung der experimentell bestimmten Avogadrokonstante zum Theoriewert ca. $9.7\%$ beträgt.



\newpage
\section{Literaturangabe}
\label{sec:Literatur}

Bilder und Daten aus dem Skript zu \emph{Millikan-Versuch}, Versuch 503, TU Dortmund, abrufbar auf:\\
\url{http://129.217.224.2/HOMEPAGE/PHYSIKER/BACHELOR/AP/SKRIPT/Millikan.pdf}\\(Stand 09.05.16)\par


\emph{Theoriewerte der physikalischen Konstanten}

NIST, \emph{Faradaykonstante}, abrufbar auf:\\
\url{http://physics.nist.gov/cgi-bin/cuu/Value?f}\\(Stand 09.05.16)\par

NIST,\emph{Avogadrokonstante}, abrufbar auf:\\
\url{http://physics.nist.gov/cgi-bin/cuu/Value?na}\\(Stand 09.05.16)\par
