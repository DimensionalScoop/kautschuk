\section{Durchführung}
\label{sec:Durchführung}

Zunächst wird die Bragg-Bedingung überprüft, indem bei festem Kristallwinkel $\Theta = 14\si{\degree}$ der Detektor von $\phi=26\si{\degree} \, .. \, 30\si{\degree}$ in $0,1\si{\degree}$ Schritten rotiert wird. Das Maximum der detektierten Intensität ist bei $2\Theta = 28\si{\degree}$ zu erwarten. Anschließend wird eine Messung ohne Absorber im Winkelbereich $4\si{\degree} \leq \Theta \leq 26\si{\degree}$ in $0,2\si{\degree}$ -Schritten durchgeführt, um das Emissionsspektrum der Cu-Röntgenröhre aufzunehmen. Schließlich werden die Spektren der Elemente aufgenommen. Dazu wird ein geeigneter Winkelbereich um die aus den Übergangsenergien berechneten Glanzwinkel eingestellt. Von besonderem Interesse für die leichten Elemente ist der $K_\alpha$ -Übergang, aus dessen Energiewert gemäß Gleichung \eqref{eq:sigma_k} die Abschirmkonstante $\sigma_K$ berechnet werden kann. Für Gold wird die $L$ -Kante ausgewählt. Hierbei werden 2 Kanten in unmittelbarer Nähe erwartet, aus deren Energiedifferenz sich gemäß Gleichung \eqref{eq:sigma_l} die Abschirmungskonstante $\sigma_L$ ergibt. Die Integrationszeit wird von vormals 5s auf 20s angehoben, um der verringerten Strahlungsdosis entgegenzuwirken.
