\section{Diskussion}
\label{sec:Diskussion}

Die Überprüfung der Braggbedingung, in Kapitel $\ref{sec:Überprüfung der Bragg Bedingung}$, ergibt einen Kristallwinkel von $\theta = 13,8 \si{\degree}$. Der zuvor eingestellte Winkel betrug $\theta_\textrm{eing.} = 14,0 \si{\degree}$. Die resultierende relative Abweichung von $1,34 \%$ ist auf eine nicht exakte Kalibrierung  des Geräts zurückzuführen. \\
Bei der Ermittlung der maximalen Energie des Bremsspektrums, in Kapitel $\ref{sec:Grenzwinkel}$, ist eine Abweichung um $ 6,54\%$ zu dem vorab eingestellten Wert von $E_\textrm{eing.} = 35 \si{\kilo\electronvolt}$ festzustellen. Die Abweichung  lässt sich ebenfalls auf die nicht exakte Kalibrierung  des Geräts zurückführen.\\
Desweiteren folgt, nach Kapitel $\ref{sec:Halbwertsbreite}$, aus der Halbwertsbreite des Emissionsspektrums, ein Auflösungsvermögen von der verwendeten Apperatur von $\input{build/Energieaufloesung_alpha.tex}$. Da das Auflösungsvermögen so gering ist, ist es nicht sinnvoll den statistischen Fehler zu bestimmen.\\
In der folgenden Tabelle werden die in Kapitel $\ref{sec:Abschirmkonstante}$  und $\ref{sec:Das Absorptionsspektrum}$ berechneten Absorptionsenergien und Abschirmkonstanten mit den Literaturwerte verglichen.


\begin{table}
  \centering
  \begin{tabular}{lSSSS}
    \toprule
    & {$E_\textrm{Lit.} \:/\: \si{\kilo\electronvolt}$} & {$E_\textrm{ermittelt} \:/\: \si{\kilo\electronvolt}$} & $\sigma_\textrm{Lit} $ & $\sigma_\textrm{ermittelt} $ \\
    \midrule
    Cu (Kupfer)        &  8.05  & $\input{build/Absorptionsenergie_Kupfer.tex} $         & $\input{build/sigma_1_lit.tex}$ & $\input{build/sigma_1.tex}$ \\
    Ge (Germanium)     &  10.9  & $\input{build/Absorptionsenergie_Germanium_ohne.tex} $ & 3.68 & $\input{build/Abschirmkonstante_Germanium.tex}$ \\
    Zn (Zink)          &  9.65  & $\input{build/Absorptionsenergie_Zink_ohne.tex}$       & 3.56 & $\input{build/Abschirmkonstante_Zink.tex}$ \\
    Zr (Zirkonium)     &  16.89 & $\input{build/Absorptionsenergie_Zirkonium_ohne.tex} $ & 4.09 & $\input{build/Abschirmkonstante_Zirkonium.tex}$ \\
    \bottomrule
  \end{tabular}
  \caption{Vergleich der Absorptionsenergien und Abschirmkonstanten.}
  \label{tab:4}
\end{table}

Über die Kante des Absorber aus Gold kann keine Aussage getroffen werden, da in Abbildung $\ref{fig:plot9}$ keine eindeutige Kante zu erkennen ist. Somit kann ebenfalls keine Aussagen über die  Absorptionsenergie und die  Abschirmkonstante getroffen werden.\\
Zu der in Kapitel \ref{sec:Moseleysches Gesetz} berechneten Rydbergonstante ist eine Abweichung von ca. $2\%$ zu dem Literaturwert $R_\infty  = 10973731.56 \si{\per\meter}$ festzustellen.\\
Die Abweichungen, bezüglich der Absorber und der Rydbergonstante, lassen sich auf eine nicht exakte Kalibrierung  des Geräts zurückzuführen.
