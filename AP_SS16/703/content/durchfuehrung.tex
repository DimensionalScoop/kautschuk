\section{Durchführung}
\label{sec:Durchführung}
Als Strahlungsquelle wird im folgenden Versuch ein Thallium-Isotop verwendet.

\subsection{Aufnahme der Charakteristik des Zählrohrs}
Der $\beta$-Strahler wird vor dem Geiger-Müller-Zählrohr positioniert, um die Charakteristik des Strahlers aufzunehmen. Dazu wird die Anzahl der eingehenden Teilchen in Abhängigkeit der angelegten Spannung $U$ gemessen. Im Bereich von $\SI{320}{\volt}$-$\SI{700}{\volt}$ wird im Abstand von $\SI{10}{\volt}$ die dazugehörige Teilchenzahl $\Delta N$ notiert.
Bei der Durchführung ist darauf zu achten, die Spannung von $\SI{720}{\volt}$ nicht zu überschreiten, da sonst durch Dauerentladungen das Zählrohr zerstört wird.

\subsection{Sichtbarmachung von Nachentladungen}
In diesem Teil soll die Nachentladung rein qualitativ gezeigt werden. Dafür wird der Abstand des $\beta$-Strahlers zum Zählrohr vergrößert, bis auf dem Oszilloskop kein weiterer Impuls des Strahlers mehr sichtbar ist. Das entstehende Bild soll einmal bei einer Spannung von $\SI{350}{\volt}$ untersucht werden, da in diesem Bereich Nachentladungen sehr unwahrscheinlich sind und einmal bei $\SI{700}{\volt}$.

\subsection{Oszillographische Messung der Totzeit}
Es wird die Strahlungsintensität erhöht indem der $\beta$-Strahler nah am Zählrohr positioniert wird. Auf dem Oszilloskop wird die Totzeit $T$ und die Erholungszeit $T_\textrm{E}$ grob abgeschätzt.


\subsection{ Bestimmung der Totzeit mit der Zwei-Quellen-Methode}
\label{sec:Totzeit}
Zuerst wird die erste Strahlungsquelle vor dem Zählrohr platziert und die Zählrate $N_1$ aufgenommen. Dann wird ein zweiter Strahler vor dem Zählrohr platziert und die Zählrate $N_\textrm{1+2}$ aufgenommen. Zuletzt wird die erste Quelle wieder entnommen um die Zählrate $N_2$ auf zu nehmen.\\
Aufgrund der Totzeit gilt
\begin{align}
  N_\textrm{1+2} < N_\textrm{1} + N_\textrm{2}\;.
  \label{eq:N1und2}
\end{align}
Auf Grundlage von Gleichung $\ref{eq:N}$ wird die Totzeit $T$ bestimmt durch
\begin{align}
  T \approx \frac{N_\textrm{1} + N_\textrm{2} - N_\textrm{1+2}}{2N_\textrm{1} N_\textrm{2}} \;.
  \label{eq:T}
\end{align}

\subsection{Messung der pro Teilchen vom Zählrohr freigesetzten Ladungsmenge }
Mit einem geeigneten Amperemeter wird der mittlere Zählrohrstrom
\begin{align}
  \bar{I} \coloneqq \frac{1}{\tau} \int_0^\tau \frac{U(t)}{R} dt
  \label{eq:Strom1}
\end{align}
wobei $\tau >> T$, gemessen.
Da die Anzahl der Teilchen $Z$ und das Zeitintervall $\Delta t$, in dem die Teilchenzahl bestimmt wird, bekannt ist, ergibt sich mit der Definition des Stroms
\begin{align}
  \bar{I} = \frac{\Delta Q}{\Delta t} Z\;.
  \label{eq:Strom2}
\end{align}
Auf Grund der Abhängigkeit der Ladungsmenge von der Zählrohrspannung, wird $\Delta Q$ während der Aufnahme der Charakteristik des Zählrohres notiert.
