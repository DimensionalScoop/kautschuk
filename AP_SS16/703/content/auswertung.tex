\section{Auswertung}
\label{sec:Auswertung}
Die im Folgenden durchgeführten Ausgleichsrechnung wird mit der \emph{curve fit} Funktion aus dem für \emph{Python} geschriebenen package \emph{NumPy}\cite{scipy} durchgeführt. Fehlerrechnungen werden mit dem für \emph{Python} geschriebenen package \emph{Uncertainties}\cite{uncertainties} ausgeführt.

Zunächst soll die Charakteristik, das heißt die Plateausteigung, des Geiger-Müller-Zählrohrs bestimmt werden. Die zu diesem Zwecke aufgenommenen Messdaten sind in Tabelle \ref{table:a} aufgelistet. Für die poissonverteilten Messwerte für die Anzahl der detektierten Impulse $Z$ gilt
\begin{equation}
  \delta Z = \sqrt{Z} \; .
\end{equation}
Zusätzlich in die Tabelle eingetragen sind die abgeleiteten Größen der pro Teilchen freigesetzten Ladungsmenge $\Delta Q$ sowie die Zählrate $N$. Die Ladungsmenge ergibt sich dabei direkt aus Gleichung \eqref{eq:Strom2}, während die Zählrate gemäß
\begin{equation}
  N = \frac{Z}{\Delta t}
\end{equation}
gegeben ist. Für beide Berechnungen ist die im Versuch eingestellte Messzeit $\Delta t = \SI{10}{\second}$ einzusetzen. Die so gewonnen Werte für die Zählrate sind in Abbildung \ref{fig:plot} gegen die Betriebsspannung des Zählrohrs aufgetragen.
\input{build/Tabelle_a_texformat}
\begin{figure}
  \centering
  \includegraphics{build/aufgabenteil_a_plot.pdf}
  \caption{Charakteristik des Zählrohrs. Die für den Plateaubereich ausgewählten Messwerte sind blau gezeichnet.}
  \label{fig:plot}
\end{figure}
Zusätzlich eingezeichnet ist eine Regressionsgerade der Form
\begin{equation*}
  N(U) = aU +b \; ,
\end{equation*}
welche die in blau eingezeichneten, für den Plateaubereich ausgewählten Messwerte berücksichtigt. Hierbei ergibt sich
\begin{align*}
    a &= \input{build/parameter_a.tex} \\
    b &= \input{build/parameter_b.tex} \; .
\end{align*}
Die Plateausteigung $s$ entspricht zwar dem Parameter $a$, jedoch wird sie zumeist in $\%/100\si{\volt}$ angegeben. Der Prozentwert soll sich dabei auf den Funktionswert in der Mitte des Plateaus $N (500\si{\volt}) = \input{build/plateaumitte.tex}$ beziehen. Dann ist
\begin{align*}
  s &= \input{build/plateausteigung.tex}(100\si{\volt})^{-1} \; .
\end{align*}
Die Länge des Plateaus wird vergleichsweise willkürlich mit
\begin{equation*}
  L = ((650-350)\pm 40)\si{\volt} = 300\pm 40 \si{\volt}
\end{equation*}
abgelesen. Der zugehörige Fehler wird aufgrund der ohnehin schwankenden Messwerte großzügig angesetzt.

Um den Zusammenhang zwischen der am Zählrohrdraht gesammelten Ladung pro eintreffendem Impuls $\Delta Q$ und der Betriebsspannung sichtbar zu machen, sind die Messwerte aus Tabelle \ref{table:a} in der Abbildung \ref{fig:plot2} visualisiert.
\begin{figure}
  \centering
  \includegraphics{build/aufgabenteil_e_plot.pdf}
  \caption{Ladung pro eintreffendem Impuls gegenüber der Betriebsspannung.}
  \label{fig:plot2}
\end{figure}
Damit ist die Charakteristik bestimmt und es verbleiben die Aussagen zu Totzeit und Erholungszeit des Geiger-Müller-Zählrohrs. Die Tabelle \ref{table:c} listet hierzu die am Oszilloskop beobachteten Messwerte auf. Zusätzlich werden die in $\si{\centi\meter}$ gemessenen Daten mit Hilfe des am Oszilloskop eingestellten Skalierungsfaktors in die entsprechenden Zeiten übersetzt.
\input{build/Tabelle_c_texformat}
Es ist anzumerken, dass auch hier ein großzügiger Fehler angenommen wird, vor allem bei der Bestimmung der Erholungszeit. Dies ist durch die in sehr unregelmäßiger Abfolge und Position auf dem Oszilloskop erscheinenden Nachentladungsimpulse zu erklären.
Zum Vergleich wird nun die Totzeit mit Hilfe der Zwei-Quellen-Methode gemäß Gleichung \eqref{eq:T} bestimmt. Das in Kapitel \ref{sec:Totzeit} beschriebene Verfahren liefert die in Tabelle \ref{table:d} zu sehenden Ergebnisse.
\input{build/Tabelle_d_texformat}
Da die wahre Größe der Totzeit nicht bekannt ist, wird der folgende relative Fehler auf den als fehlerfrei angenommenen Mittelwert $\overline{T}$ der Messungen der zwei verschiedenen Methoden bezogen.
\begin{align*}
  \epsilon_\text{Tot} = \frac{T_\text{Oszi} - T_\text{2-Quell}} {\overline{T}} = \input{build/RelFehler.tex}
\end{align*}

% Sämtliche im Folgenden durchgeführten Ausgleichsrechnungen werden mit der \emph{curve fit} Funktion aus dem für \emph{Python} geschriebenen package \emph{NumPy}\cite{scipy} durchgeführt. Fehlerrechnungen werden mit dem für \emph{Python} geschriebenen package \emph{Uncertainties}\cite{uncertainties} ausgeführt.

% % Examples
% \begin{equation}
%   U(t) = a \sin(b t + c) + d
% \end{equation}
%
% \begin{align}
%   a &= \input{build/a.tex} \\
%   b &= \input{build/b.tex} \\
%   c &= \input{build/c.tex} \\
%   d &= \input{build/d.tex} .
% \end{align}
% Die Messdaten und das Ergebnis des Fits sind in Abbildung~\ref{fig:plot} geplottet.
%
% %Tabelle mit Messdaten
% \begin{table}
%   \centering
%   \caption{Messdaten.}
%   \label{tab:data}
%   \sisetup{parse-numbers=false}
%   \begin{tabular}{
% % format 1.3 bedeutet eine Stelle vorm Komma, 3 danach
%     S[table-format=1.3]
%     S[table-format=-1.2]
%     @{${}\pm{}$}
%     S[table-format=1.2]
%     @{\hspace*{3em}\hspace*{\tabcolsep}}
%     S[table-format=1.3]
%     S[table-format=-1.2]
%     @{${}\pm{}$}
%     S[table-format=1.2]
%   }
%     \toprule
%     {$t \:/\: \si{\milli\second}$} & \multicolumn{2}{c}{$U \:/\: \si{\kilo\volt}$\hspace*{3em}} &
%     {$t \:/\: \si{\milli\second}$} & \multicolumn{2}{c}{$U \:/\: \si{\kilo\volt}$} \\
%     \midrule
%     1.7 & 10 \\
2.3 & 20 \\
3.5 & 30 \\
4.4 & 40 \\

%     \bottomrule
%   \end{tabular}
% \end{table}
%
% % Standard Plot
% \begin{figure}
%   \centering
%   \includegraphics{build/plot.pdf}
%   \caption{Messdaten und Fitergebnis.}
%   \label{fig:plot}
% \end{figure}
%
% 2x2 Plot
% \begin{figure*}
%     \centering
%     \begin{subfigure}[b]{0.475\textwidth}
%         \centering
%         \includegraphics[width=\textwidth]{Abbildungen/Schaltung1.pdf}
%         \caption[]%
%         {{\small Schaltung 1.}}
%         \label{fig:Schaltung1}
%     \end{subfigure}
%     \hfill
%     \begin{subfigure}[b]{0.475\textwidth}
%         \centering
%         \includegraphics[width=\textwidth]{Abbildungen/Schaltung2.pdf}
%         \caption[]%
%         {{\small Schaltung 2.}}
%         \label{fig:Schaltung2}
%     \end{subfigure}
%     \vskip\baselineskip
%     \begin{subfigure}[b]{0.475\textwidth}
%         \centering
%         \includegraphics[width=\textwidth]{Abbildungen/Schaltung4.pdf}    % Zahlen vertauscht ... -.-
%         \caption[]%
%         {{\small Schaltung 3.}}
%         \label{fig:Schaltung3}
%     \end{subfigure}
%     \quad
%     \begin{subfigure}[b]{0.475\textwidth}
%         \centering
%         \includegraphics[width=\textwidth]{Abbildungen/Schaltung3.pdf}
%         \caption[]%
%         {{\small Schaltung 4.}}
%         \label{fig:Schaltung4}
%     \end{subfigure}
%     \caption[]
%     {Ersatzschaltbilder der verschiedenen Teilaufgaben.}
%     \label{fig:Schaltungen}
% \end{figure*}
