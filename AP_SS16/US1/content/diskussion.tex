\section{Diskussion}
\label{sec:Diskussion}
Die Messung der Schallgeschwindigkeit mittels Impuls-Echo-Verfahren liefert einen Wert von
\begin{align*}
  c_{\text{acryl,e}} &= \input{build/c_acryl_1.tex}.
\end{align*}
Daraus folgt eine Abweichung zum Literaturwert \cite{acryl},
\begin{align*}
  c_{\text{acryl,lit}} &= \input{build/v_lit.tex},
\end{align*}
von
\begin{align*}
  \increment c_{\text{acryl}} &= \input{build/v_rel_1.tex}.
\end{align*}
Bei der Durchschallungs-Methode ergibt sich für die Phasengeschwindigkeit in Acryl
\begin{align*}
  c_{\text{acryl,d}} &= \input{build/c_acryl_2.tex},
\end{align*}
die relative Abweichung zum Literaturwert beträgt hier
\begin{align*}
  \increment c_{\text{acryl,d}} &= \input{build/v_rel_2.tex}.
\end{align*}
Die dritte Bestimmung der Schallgeschwindigkeit erfolgte über die Steigung der Ausgleichsrechnung, hier ergab sich ein Wert von
\begin{align*}
  c_{\text{acryl, a}} &= \input{build/parameter_a.tex}
\end{align*}
mit einer relativen Abweichung von
\begin{align*}
  \increment c_{\text{acryl,a}} &= \input{build/v_rel_3.tex}.
\end{align*}
Die Impuse-Echo-Methode erweist sich dementsprechend als exakteste.\\
Die Anpassungsschicht wurde mittels y-Achsenabschnitt der Ausgleichsrechnung zu
\begin{align*}
  d &= \input{build/parameter_b.tex}.
\end{align*}
Dieser Wert scheint eine gute Beschreibung der Anpassungsschicht zu sein, da laut Literatur eine Dicke von durchschnittlich $\SI{0.5}{\milli\metre}$ bis zu $\SI{2.5}{\milli\metre}$ zu erwarten ist. \cite{anpassungsschicht}\\
Bei der Ausmessung der Plattendicken mittels Cepstrum und FFT konnte jeweils nur eine Länge
\begin{align*}
  s_\text{fft} &= \input{build/s_probe.tex}, \\
  s_\text{cep} &= \input{build/s_cep.tex},
\end{align*}
unbekannter Bedeutung bestimmt werden.
Vermutlich handelt es sich hierbei um die kombinierte Dicke beider Platten.\\
Die Untersuchung des Augenmodelles führte auf die Längen
\begin{align*}
  s_{1} &= \input{build/s_12.tex}, \\
  s_{2} &= \input{build/s_23.tex}, \\
  s_{3} &= \input{build/s_36.tex}, \\
\end{align*}
welche ungefähr die Größe skalierte Größe eines menschlichen Auges beschreiben.
