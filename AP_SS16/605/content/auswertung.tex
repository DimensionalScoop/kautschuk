\section{Auswertung}
\label{sec:Auswertung}
Für die folgenden Berechnungen ist es aufgrund der Verwendung eines Reflexionsgitters zunächst notwendig, aus den aufgenommenen Winkeln die tatsächlichen Beugungswinkel abzuleiten. Dazu dienen die Ausführungen in Kapitel \ref{sec:winkelbeziehungen}. Das ungebeugte Maximum nullter Ordnung wird gerade unter dem Einfallswinkel festgestellt, denn für Reflexion gilt Ausfallwinkel gleich Einfallwinkel. Die Bezeichnungen der Winkel orientieren sich an denen aus Abb. \ref{fig:winkel}. Der gemessene Reflexionswinkel beträgt
\begin{equation}
  \delta = 338,2\si{\degree}  \;\;\;\; \Rightarrow \;\;\;\; \beta = 59,1\si{\degree}
\end{equation}
sodass sich für den zu ermittelnden Beugungswinkel $\varphi$ mit dem an der Skala abgelesenen Winkel $\delta$'
\begin{equation}
  \varphi = 279,1\si{\degree} - \delta\text{'}
  \label{eq:deltaZuPhi}
\end{equation}
ergibt. Die Messdaten zeigen, dass die gemessenen Beugungswinkel allesamt negativ sind. Im Folgenden werden die Winkel jedoch der Übersicht halber mit ihrem Betrag angegeben, während für die Rechnungen das Minuszeichen beibehalten wird. Ferner werden die Winkel in den folgenden Unterkapiteln ins Bogenmaß übersetzt. Tabelle \ref{tab:messdaten_winkel} zeigt die gemessenen und mit Gleichung \eqref{eq:deltaZuPhi} umgerechneten Beugungswinkel für Natrium, Kalium und Rubidium.
\begin{table}
  \centering
  \caption{Gemessene und umgerechnete Beugungswinkel von drei verschiedenen Alkali Metallen.}
  \label{tab:messdaten_winkel}
  \sisetup{parse-numbers=false}
  \begin{tabular}{
    S[table-format=2.1]
    S[table-format=2.1]
    S[table-format=2.1]
  }
    \toprule
    {$|\varphi_\text{Na}| \:/\: \si{\degree}$} & {$|\varphi_\text{K}| \:/\: \si{\degree}$} &
    {$|\varphi_\text{Rb}| \:/\: \si{\degree}$} \\
    \midrule
    11.6 & 14.1 & 7.9 \\
11.5 & 14.0 & 7.3 \\
10.2 & 14.0 & \\
10.1 & 13.9 & \\
8.4  & 11.0 & \\
8.3  & 10.8 & \\
     & 10.7 & \\
     & 10.6 & \\

    \bottomrule
  \end{tabular}
\end{table}
Sämtliche im Folgenden durchgeführten Ausgleichsrechnungen werden mit der \emph{curve fit} Funktion aus dem für \emph{Python} geschriebenen package \emph{NumPy}\cite{scipy} durchgeführt. Fehlerrechnungen werden mit dem für \emph{Python} geschriebenen package \emph{Uncertainties}\cite{uncertainties} ausgeführt.

\subsection{Bestimmung der Gitterkonstante}
\label{sec:gitterkonstante}
Aus den gemessenen Beugungswinkeln bekannter Spektrallinien des Heliums wird im Folgenden auf die Gitterkonstante geschlossen. Die Tabelle \ref{table:gitterkonstante} listet die zu den jeweils gegebenen Wellenlängen die gemessenen Winkel sowie deren Sinus auf.
\input{build/Tabelle_a_texformat.tex}
Eine lineare Regressionsrechnung liefert für diesen Datensatz gemäß Gleichung \eqref{eq:sinphi} mit $k=1$ dem ersten Begungsmaximum die Gitterkonstante als Proportionalität zwischen Wellenlänge und dem Sinus des Beugungswinkels. Mit der Geradengleichung
\begin{equation*}
  f(x) = mx+b
\end{equation*}
ergibt sich
\begin{align*}
  m = g = \input{build/gitterkonstante.tex}\\
  b = \input{build/offset.tex}
\end{align*}
Der Graph der auf diese Weise ermittelten Regressionsgeraden ist samt den aufgenommenen Messwerten in Abbildung \ref{fig:linreg} wiedergegeben.
\begin{figure}
  \centering
  \includegraphics{build/aufgabenteil_a_plot.pdf}
  \caption{Messdaten und Regressionsgerade zur Bestimmung der Gitterkonstanten $g$.}
  \label{fig:linreg}
\end{figure}

\subsection{Kalibrierungsgröße}
Für eine präzisere Messung der Wellenlängendifferenz $\Delta \lambda$ zwischen zwei Dublettlinien ist die Okularlinse zum Einsatz gekommen. Die auf der Linse angebrachte Skala wird zusammen mit dem Okularmikrometer dazu verwendet, die Winkelauflösung zu erhöhen. Dazu wird jedoch zunächst eine Referenzmessung benötigt, die sich aus jeweils zwei im Spektrum des Heliums dicht beieinander liegender Wellenlängen ergibt. Für die Umrechnung der gemessenen Skalendifferenz $\Delta s$ wird gemäß \eqref{eq:deltalambda} der Kalibrierungsterm
\begin{equation*}
  \chi \coloneq \frac{\lambda_1 - \lambda_2}{t_1 - t_2}\overline{\varphi}_{1,2}
\end{equation*}
benötigt. Die gegebenen Wellenlängen, zugehörigen Winkel und die mit dem Okularmikrometer gemessenen Skalenwerte $t$ sind in Tabelle \ref{table:eichgroesse} wiedergegeben. Die Einheit Skt bezeichnet die im Okularmikrometer zu sehende Skala.
\input{build/Tabelle_b_texformat.tex}
Hieraus errechnen sich zwei Werte für $\chi$, aus denen sich der Mittelwert
\begin{equation*}
  \chi = \input{build/Eichgroesse.tex}
\end{equation*}
ergibt. Der angegebene Fehler ist jedoch mit Vorsicht zu genießen, denn es sind lediglich zwei Messwert-Paare aufgenommen worden.

\subsection{Abschirmungszahlen}
Von besonderem Interesse sind Dublettlinien von Alkali-Metallen, aus deren Vermessung sich die inneren Abschirmungszahlen $\sigma_2$ ergeben, vgl. Gleichung \eqref{eq:sigma2}. Hierzu wird die Übergangsenergie $E_D$ benötigt. Sie ergibt sich aus Gleichung \eqref{eq:E_D} unter Kenntnis der Wellenlängendifferenz $\Delta \lambda$ der beiden beteiligten Dublettlinien sowie deren mittlerer Wellenlänge $\overline{\lambda}$. Die Differenz $\Delta \lambda$ wird mit Hilfe des Okularmikrometers deutlich präziser bestimmt als durch die vergleichsweise grobe Messung mit Hilfe des Teilkreises. Es folgt unter Kenntnis der Kalibrierungsgröße $\chi$ aus dem letzten Unterkapitel und Gleichung \eqref{eq:deltalambda}
\begin{equation*}
  \Delta\lambda = \chi \cos(\overline{\varphi}) \Delta s  \; .
\end{equation*}
Die gemessenen Skalendifferenzen $\Delta s$ sind ebenso wie die restlichen benötigten Größen für die Elemente Natrium, Kalium und Rubidium in den Tabellen \ref{table:natrium}-\ref{table:rubidium} angegeben. Die Wellenlängen berechnen sich aus der Regressionsgeraden aus Kapitel \ref{sec:gitterkonstante} und den einzusetzenden gemessenen Beugungswinkeln aus Tabelle \ref{tab:messdaten_winkel}. Die daraus resultierenden Mittelwerte sind ebenfalls in den unten stehenden Tabellen zu sehen.

Schließlich werden für jedes Dublett die Übergangsenergien $E_D$ in Gleichung \eqref{eq:sigma2} eingesetzt, um die Abschirmungszahlen zu berechnen. Da Übergänge der äußeren Schale betrachtet werden, sind für die Hauptquantenzahl $n$ die Perioden der jeweiligen Elemente einzusetzen. Die Ordnungszahlen können ebenso direkt dem Periodensystem der Elemente entnommen werden. Die Nebenquantenzahl $l$ ist in \cite{skript} jeweils mit 1 angegeben.
\input{build/Tabelle_c_natrium_texformat.tex}
\input{build/Tabelle_c_kalium_texformat.tex}
\input{build/Tabelle_c_rubidium_texformat.tex}
Die Abschirmungszahlen für Natrium und Kalium werden aus den Resultaten der vermessenen Dublettlinien gemittelt und der statistische Fehler des Mittelwertes angegeben. Der systematische Fehler ist deutlich kleiner, vergleiche hierzu die Fehlerwerte für $\sigma_2$ in den Tabellen. Daher ist es sinnvoll, die statistischen Fehler nicht mit anzugeben. Für Rubidium hingegen ergibt sich aus einer einzigen Messung kein Mittelwertfehler, sodass hier der systematische Fehler angegeben wird.
\begin{align*}
  \sigma_{2,\text{Na}} &= \input{build/sigma_natrium.tex}  \\
  \sigma_{2,\text{K}} &= \input{build/sigma_kalium.tex}    \\
  \sigma_{2,\text{Rb}} &= \input{build/sigma_rubidium.tex} \\
\end{align*}
% % Examples
% \begin{equation}
%   U(t) = a \sin(b t + c) + d
% \end{equation}
%
% \begin{align}
%   a &= \input{build/a.tex} \\
%   b &= \input{build/b.tex} \\
%   c &= \input{build/c.tex} \\
%   d &= \input{build/d.tex} .
% \end{align}
% Die Messdaten und das Ergebnis des Fits sind in Abbildung~\ref{fig:plot} geplottet.
%
% %Tabelle mit Messdaten
% \begin{table}
%   \centering
%   \caption{Messdaten.}
%   \label{tab:data}
%   \sisetup{parse-numbers=false}
%   \begin{tabular}{
% % format 1.3 bedeutet eine Stelle vorm Komma, 3 danach
%     S[table-format=1.3]
%     S[table-format=-1.2]
%     @{${}\pm{}$}
%     S[table-format=1.2]
%     @{\hspace*{3em}\hspace*{\tabcolsep}}
%     S[table-format=1.3]
%     S[table-format=-1.2]
%     @{${}\pm{}$}
%     S[table-format=1.2]
%   }
%     \toprule
%     {$t \:/\: \si{\milli\second}$} & \multicolumn{2}{c}{$U \:/\: \si{\kilo\volt}$\hspace*{3em}} &
%     {$t \:/\: \si{\milli\second}$} & \multicolumn{2}{c}{$U \:/\: \si{\kilo\volt}$} \\
%     \midrule
%     1.7 & 10 \\
2.3 & 20 \\
3.5 & 30 \\
4.4 & 40 \\

%     \bottomrule
%   \end{tabular}
% \end{table}
%
% % Standard Plot
% \begin{figure}
%   \centering
%   \includegraphics{build/plot.pdf}
%   \caption{Messdaten und Fitergebnis.}
%   \label{fig:plot}
% \end{figure}
%
% 2x2 Plot
% \begin{figure*}
%     \centering
%     \begin{subfigure}[b]{0.475\textwidth}
%         \centering
%         \includegraphics[width=\textwidth]{Abbildungen/Schaltung1.pdf}
%         \caption[]%
%         {{\small Schaltung 1.}}
%         \label{fig:Schaltung1}
%     \end{subfigure}
%     \hfill
%     \begin{subfigure}[b]{0.475\textwidth}
%         \centering
%         \includegraphics[width=\textwidth]{Abbildungen/Schaltung2.pdf}
%         \caption[]%
%         {{\small Schaltung 2.}}
%         \label{fig:Schaltung2}
%     \end{subfigure}
%     \vskip\baselineskip
%     \begin{subfigure}[b]{0.475\textwidth}
%         \centering
%         \includegraphics[width=\textwidth]{Abbildungen/Schaltung4.pdf}    % Zahlen vertauscht ... -.-
%         \caption[]%
%         {{\small Schaltung 3.}}
%         \label{fig:Schaltung3}
%     \end{subfigure}
%     \quad
%     \begin{subfigure}[b]{0.475\textwidth}
%         \centering
%         \includegraphics[width=\textwidth]{Abbildungen/Schaltung3.pdf}
%         \caption[]%
%         {{\small Schaltung 4.}}
%         \label{fig:Schaltung4}
%     \end{subfigure}
%     \caption[]
%     {Ersatzschaltbilder der verschiedenen Teilaufgaben.}
%     \label{fig:Schaltungen}
% \end{figure*}
