\section{Durchführung}
\label{sec:Durchführung}

\subsection{Strömungsgeschwindigkeit}
Es wird für fünf verschiedene Geschwindigkeiten zwischen $50\%-60\%$ die Frequenzverschiebung $\Delta v$ bei jeweils allen drei Winkeln des Prismas gemessen und notiert. Diese Messung wird für alle drei Prismen und somit für alle drei Rohrdurchmesser wiederholt.
Damit sich keine Luftschicht zwischen Ultraschallsonde und Prisma bzw. Prisma und Rohrmantel ausbildet, wird ein Kontaktmittel aufgetragen, damit die Schallintensität so wenig wie möglich gedämpft wird.




\subsection{Strömungsprofil der Dopplerflüssigkeit}
Zu Beginn wird das $\emph{sample volume}$ beim Ultraschallgenerator auf $\emph{small}$ gestellt. Die Messung wird an dem Systemabschnitt durchgeführt, an dem der Rohrdurchmesser $\SI{10}{\milli\meter}$ beträgt, zudem wird nur die Prismafläche verwendet, die einen Winkel von $15 ^\circ$ zur Senkrechten hat.
Für die Messung wird die Tiefe schrittweise von $4$µs bis $19$µs mit dem Regler $\emph{Depth}$ erhöht. Für jeden Schritt wird die Frequenzdifferenz und die Streuintensität notiert.\\
Dieses Messverfahren wird für eine Pumpgeschwindigkeit von $45\%$ und $70\%$ durchgeführt.
