\section{Diskussion}
\label{sec:Diskussion}
Ein Vergleich der Geschwindigkeiten aus den Messungen mit verschiedenen Prismenwinkeln offenbart, dass die Messwerte des 60° Winkels für den Innendurchmesser $d_i = \SI{16}{\milli\meter}$ tendenziell kleiner sind als diejenigen aus den Messungen für 15° und 30°. Dieses Verhalten ist für die anderen beiden Innendurchmesser nicht festzustellen. Hier schwanken die Werte mit einer statistischen Dynamik, ohne dabei eine Tendenz zu zeigen. Ein möglicher Grund könnte in dem Strömungsprofil zu finden sein, denn für größere Innendurchmesser sollte die Schallwelle vor Allem die Frequenzverschiebung aus der Rohrmitte zeigen, wo die Geschwindigkeit ihren Maximalwert annimmt. Für einen Prismenwinkel von 60° ist dies möglicherweise nicht mehr gewährleistet, sodass tendenziell die kleineren Geschwindigkeiten aus dem Bereich der Rohrwände gemessen werden.

Die Abbildung \ref{fig:Geschwindigkeiten} zeigt, dass die Geschwindigkeiten im Allgemeinen mit dem Rohrdurchmesser abnehmen. Dies ist das erwartete Verhalten wie es von dem Gesetz von Bernoulli vorhergesagt wird. Ein kleinerer Durchmesser bei gleicher Druckdifferenz führt im laminaren Bereich zu höheren Strömungsgeschwindigkeiten. Ebenfalls der erwartete Anstieg der Geschwindigkeiten mit der Pumpleistung kann festgestellt werden.

Das Strömungsprofil, welches mit Hilfe der tiefenabhängigen Messung bestimmt worden ist, weist einen näherungsweise parabelförmigen Verlauf auf, sodass auch hier die Erwartungen gemäß dem Gesetz von Hagen-Poiseuille bestätigt werden können. Die entsprechenden Fits in Abbildung \ref{fig:profil} zeigen dieses Profil. An den Rohrwänden nimmt die Geschwindigkeit ab. Die Messungen der Streuintensitäten bestätigen die vermuteten Grenzübergänge zwischen Prismenmaterial -- Rohr -- Dopplerflüssigkeit nur ansatzweise. Während für eine Pumpleistung von 45\% ein allgemeiner Anstieg der Intensität ab ca. $\SI{12}{\micro\second}$ zu sehen ist, lässt sich für eine Pumpleistung von 70\% noch eine gewisse Schwankung erkennen.

Während das reflektierende Ende des Rohres deutlich zu erkennen ist durch einen steilen Anstieg im Verlauf der Intensitäten bei ca. $\SI{19}{\micro\second}$ kann die erste Grenzschicht bei ca. $\SI{12}{\micro\second}$ nur durch einen schwachen Anstieg identifiziert werden.

Die erhöhte Intensität im Bereich der Strömung ist dadurch zu erklären, dass hier Reflexionen mit den Glaskörpern der Dopplerflüssigkeit stattfinden. Unklar ist jedoch, wieso im Bereich des Prismenmaterials links von $\SI{12}{\micro\second}$ eine Frequenzverschiebung zu sehen ist. Ihr Wert entspricht näherungsweise dem Mittelwert der Strömungsgeschwindigkeit, sodass die Vermutung nahe liegt, dass sich die Messung durch das Einstellen der Laufzeit nicht in solch einfacher Weise von der Strömung trennen lässt.
