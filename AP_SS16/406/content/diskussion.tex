\section{Diskussion}
\label{sec:Diskussion}
Die Messungen der Spaltbreiten sowie des Spaltabstandes des Doppelspalts liefern erstaunlich präzise Werte, wenn man die Gültigkeit der Herstellerangaben voraussetzt. Dies Resultat haben die nur schwach erkennbaren Strukturen im Mikroskop nicht vermuten lassen. Der Rückschluss auf die Spaltabmessungen anhand der mit dem Detektor aufgenommenen Intensitätsverteilung hingegen weist deutlich größere Abweichungen auf. Die relativen Fehler zu den Herstellerangaben sind in der folgenden Tabelle erneut aufgelistet.
\input{build/Tabelle_errors_texformat.tex}
Der kleinste relative Fehler zeigt sich für den mittelgroßen Spalt. Auch der Fit zu diesem Spalt zeigt eine gute Übereinstimmung zwischen Theorie und Experiment, siehe Abbildung \ref{fig:A2}. Die anderen Messungen hingegen weisen mit ca. 20 bis $\SI{30}{\percent}$ eine vergleichsweise große Abweichung auf. Für den kleinen Einzelspalt kann anhand der Messwerte das erste Beugungsmaximum nicht gefunden werden, siehe Abbildung \ref{fig:A1}. Der Fit würde im weiteren Verlauf dieses Maximum zeigen, jedoch hat sich im Versuch durch weiteres Verfahren des Detektors kein Maximum eingestellt. Der Messwert für die Spaltbreite mit Hilfe des Fits ist an dieser Stelle also zu bezweifeln. Ein möglicher Grund könnte die geringe Intensität sein. Dafür spricht, dass der Dunkelstrom, welcher zu Beginn der Messungen aufgenommen worden ist, ohnehin in den Randbereichen bereits oberhalb der Messwerte liegt. Der Fit für den großen Einzelspalt hat den Makel, dass das erste Beugungsmaximum nicht getroffen wird, siehe Abbildung \ref{fig:A3}. Die Messwerte zeigen, dass dieses Maximum eigentlich bei kleineren Winkeln zu finden ist. Dies würde zu einem größeren Fitparameter $b$ führen und kann somit als Erklärung für den Unterschied zu der Herstellerangabe dienen. Tatsächlich wird der Fit überwiegend durch die Ausprägung des nullten Beugungsmaximum bestimmt.

Für den Doppelspalt liegt der Messwert des Spaltabstandes in einem akzeptablen Bereich. Durch ihn wird die Oszillationsfrequenz der Intensitätsverteilung bestimmt, siehe Abbildung \ref{fig:A4}. Die Spaltbreite hingegen beschreibt die Ausschmierung der Einhüllenden. Ihr Einfluss ist schematisch gut im Verlauf der Abbildungen für die Einzelspalte zu erkennen. Eine kleinere Spaltbreite lässt das nullte Beugungsmaximum ausschmieren, während eine große Spaltbreite einen scharfen Peak in der Mitte erzeugt. Der Fit der Messwerte für den Doppelspalt zeigt nun, dass zwar die Messwerte im Wesentlichen auf der Kurve liegen, jedoch im Bereich um $\varphi=0$ die kleinen Stromstärken überhaupt nicht bestätigt werden können. Im Experiment ist hier kein rapides Absinken der Stromstärke festgestellt worden. Insofern kann auch hier der Fit nur als Näherung dienen und seine Gültigkeit wiederum angezweifelt werden. Ein möglicher und primär zu nennender Grund ist die mangelnde Ortsauflösung des Detektors. Auch hier fällt das Licht durch einen Spalt endlicher Breite, sodass auch die Intensität rechts und links des scharf definierten Messpunktes aufgenommen werden.
