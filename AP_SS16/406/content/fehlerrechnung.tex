\section{Fehlerrechnung}
Dieses Kapitel listet kurz und bündig die benötigten und aus den Methoden der Statistik bekannten Formeln für die Fehlerrechnung auf.
Die Schätzung der Standardabweichung ist
\begin{equation}
  \label{eq:std}
  \Delta X = \sqrt{\frac{1}{n-1}\sum_{i=1}^n(X_i-\overline{X})^2}     \; .
\end{equation}
Der Mittelwert ist
\begin{equation}
  \overline{X} = \frac{1}{n} \sum_{i=1}^nX_i
\end{equation}
Der Fehler des Mittelwertes ist
\begin{equation}
  \label{eq:std_mean}
  \Delta \overline{X} = \sqrt{\frac{1}{n(n-1)}\sum_{i=1}^n(X_i-\overline{X})^2}   \; .
\end{equation}
Für fehlerbehaftete Größen, die auch in folgenden Formeln verwendet werden, muss die Fehlerfortpflanzung nach Gauß berücksichtigt werden.
\begin{equation}
  \label{eq:GFFP}
  \Delta f = \sqrt{\sum_{i=1}^n \left(\frac{\partial f}{\partial X_i}\right)^2 \cdot (\Delta X_i)^2}
\end{equation}
