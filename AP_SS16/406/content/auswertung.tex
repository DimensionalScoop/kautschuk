\section{Auswertung}
\label{sec:Auswertung}


Sämtliche im Folgenden durchgeführten Ausgleichsrechnungen werden mit der \emph{curve fit} Funktion aus dem für \emph{Python} geschriebenen package \emph{NumPy}\cite{scipy} durchgeführt. Fehlerrechnungen werden mit dem für \emph{Python} geschriebenen package \emph{Uncertainties}\cite{uncertainties} ausgeführt.

Zunächst werden die Messwerte (siehe Anhang) in die für die weiteren Berechnungen benötigte Form übersetzt. Dazu werden alle Positionen gemäß
\begin{equation}
  \varphi \approx \tan\varphi = \frac{\zeta - \zeta_0}{L}
\end{equation}
mit dem Abstand zwischen Detektor und Spalt $L$ in Winkel übersetzt. Der Offset $\zeta_0$ resultiert aus der nicht perfekt mittig aufgenommenen Messung. Dies wird unmittelbar aus den Abbildungen \ref{fig:A1} bis \ref{fig:A4} klar. Da $\zeta_0$ vor Allem auch für die folgenden Fits von großer Relevanz ist, sei an dieser Stelle auf eine nähere Analyse weiter unten verwiesen. Außerdem wird die notierte Skala entsprechend mit den Messwerten der Stromstärke verrechnet, um diese in die SI-Einheit Ampere zu bringen. Es ist anzumerken, dass hier ein möglicher Messfehler des Messgeräts vernachlässigt wird. Sein Einfluss lässt sich im Nachhinein durch einen etwas vergrößerten Fehler abschätzen. Üblicherweise beträgt der Messfehler des analogen Amperemeters $\SI{5}{\percent}$ der Skala. Schließlich wird der Messwert der Dunkelstrommessung subtrahiert.
\begin{equation*}
  I_\text{Dunkel} = \SI{7.1}{\nano\ampere}
\end{equation*}

Die Messwerte $\tilde{l}$, die mit Hilfe des Mikroskops generiert worden sind, werden gemäß
\begin{equation}
  l = \tilde{l} \frac{\SI{0.5}{\milli\meter}}{\SI{1.3}{\centi\meter}} \frac{3,2}{4}
\end{equation}
in die wahre Größe $l$ umgerechnet. Dies ist ein Resultat der vorangegangenen Kalibrierung, welche mit einer Vergrößerung von $3,2$ gegenüber der später verwendeten Vergrößerung von $4$ durchgeführt worden ist. Für eine bessere Übersicht sind diese Messwerte zusammen mit den Herstellerangaben und den im Folgenden ermittelten Fitparametern in den Tabellen \ref{table:A1} und \ref{table:A2} aufgelistet.
\input{build/Tabelle_results_texformat.tex}
\input{build/Tabelle_results_s_texformat.tex}
Um nun die Daten gemäß den Gleichungen \eqref{eq:ES_Intensitaet} (Einzelspalte) und \eqref{eq:DS_Intensitaet} (Doppelspalt) fitten zu können, muss zunächst der Offset $\zeta_0$ ermittelt werden. Dazu wird der Einfachheit halber $\zeta_0$ mit Hilfe der Plots abgeschätzt. Es ergeben sich die in Tabelle \ref{table:A1} angegebenen Werte. Nun werden die Messdaten durch eine nicht-lineare Ausgleichsrechnung an die Theoriekurven \eqref{eq:ES_Intensitaet} und \eqref{eq:DS_Intensitaet} angepasst. Die Fits und zugehörigen Messwerte sind in den Abbildungen \ref{fig:A1} bis \ref{fig:A4} dargestellt.
\begin{figure}
  \centering
  \includegraphics{build/plot_klein.pdf}
  \caption{Messdaten und Fitergebnis für den kleinen Einzelspalt.}
  \label{fig:A1}
\end{figure}
\begin{figure}
  \centering
  \includegraphics{build/plot_mittel.pdf}
  \caption{Messdaten und Fitergebnis für den mittelgroßen Einzelspalt.}
  \label{fig:A2}
\end{figure}
\begin{figure}
  \centering
  \includegraphics{build/plot_gross.pdf}
  \caption{Messdaten und Fitergebnis für den großen Einzelspalt.}
  \label{fig:A3}
\end{figure}
\begin{figure}
  \centering
  \includegraphics{build/plot_ds.pdf}
  \caption{Messdaten und Fitergebnis für den Doppelspalt. Zusätzlich ist auch der überlagerte Fit des Einzelspalts eingezeichnet.}
  \label{fig:A4}
\end{figure}
% % Examples
% \begin{equation}
%   U(t) = a \sin(b t + c) + d
% \end{equation}
%
% \begin{align}
%   a &= \input{build/a.tex} \\
%   b &= \input{build/b.tex} \\
%   c &= \input{build/c.tex} \\
%   d &= \input{build/d.tex} .
% \end{align}
% Die Messdaten und das Ergebnis des Fits sind in Abbildung~\ref{fig:plot} geplottet.
%
% %Tabelle mit Messdaten
% \begin{table}
%   \centering
%   \caption{Messdaten.}
%   \label{tab:data}
%   \sisetup{parse-numbers=false}
%   \begin{tabular}{
% % format 1.3 bedeutet eine Stelle vorm Komma, 3 danach
%     S[table-format=1.3]
%     S[table-format=-1.2]
%     @{${}\pm{}$}
%     S[table-format=1.2]
%     @{\hspace*{3em}\hspace*{\tabcolsep}}
%     S[table-format=1.3]
%     S[table-format=-1.2]
%     @{${}\pm{}$}
%     S[table-format=1.2]
%   }
%     \toprule
%     {$t \:/\: \si{\milli\second}$} & \multicolumn{2}{c}{$U \:/\: \si{\kilo\volt}$\hspace*{3em}} &
%     {$t \:/\: \si{\milli\second}$} & \multicolumn{2}{c}{$U \:/\: \si{\kilo\volt}$} \\
%     \midrule
%     1.7 & 10 \\
2.3 & 20 \\
3.5 & 30 \\
4.4 & 40 \\

%     \bottomrule
%   \end{tabular}
% \end{table}
%
% % Standard Plot
% \begin{figure}
%   \centering
%   \includegraphics{build/plot.pdf}
%   \caption{Messdaten und Fitergebnis.}
%   \label{fig:plot}
% \end{figure}
%
% 2x2 Plot
% \begin{figure*}
%     \centering
%     \begin{subfigure}[b]{0.475\textwidth}
%         \centering
%         \includegraphics[width=\textwidth]{Abbildungen/Schaltung1.pdf}
%         \caption[]%
%         {{\small Schaltung 1.}}
%         \label{fig:Schaltung1}
%     \end{subfigure}
%     \hfill
%     \begin{subfigure}[b]{0.475\textwidth}
%         \centering
%         \includegraphics[width=\textwidth]{Abbildungen/Schaltung2.pdf}
%         \caption[]%
%         {{\small Schaltung 2.}}
%         \label{fig:Schaltung2}
%     \end{subfigure}
%     \vskip\baselineskip
%     \begin{subfigure}[b]{0.475\textwidth}
%         \centering
%         \includegraphics[width=\textwidth]{Abbildungen/Schaltung4.pdf}    % Zahlen vertauscht ... -.-
%         \caption[]%
%         {{\small Schaltung 3.}}
%         \label{fig:Schaltung3}
%     \end{subfigure}
%     \quad
%     \begin{subfigure}[b]{0.475\textwidth}
%         \centering
%         \includegraphics[width=\textwidth]{Abbildungen/Schaltung3.pdf}
%         \caption[]%
%         {{\small Schaltung 4.}}
%         \label{fig:Schaltung4}
%     \end{subfigure}
%     \caption[]
%     {Ersatzschaltbilder der verschiedenen Teilaufgaben.}
%     \label{fig:Schaltungen}
% \end{figure*}
