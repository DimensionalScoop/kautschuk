\section{Durchführung}
\label{sec:Durchführung}

Zu Beginn werden die zu verwendenden Spalte, insgesamt drei Einzelspalte und ein Doppelspalt ausgemessen. Da das digitale Mikroskop ein vergrößerungsinvariantes Fokussierfenster hat, wird dieses mit einem Lineal vermessen und später mit der Vergrößerung verrechnet um die Spaltbreite festzustellen.\\
Danach wird eine Nullmessung durchgeführt, da die Photodiode trotz ausgeschaltetem Laser Umgebungslicht detektiert, damit diese "Grundbeleuchung" aus den später aufgenommenen Daten herausgerechnet werden kann.\\
Für jeden der Drei Einzelspalte werden $51$ Messpunkte aufgenommen. Die Photodiode wird solange verschoben bis sie das größte Maxima detektiert. Von da aus werden $25$ Messwerte nach links und nach rechts  aufgenommen.\\
Für den Doppelspalt werden insgesamt $81$ Messwerte aufgenommen, wieder werden von der Mitte aus $40$ Messwerte nach links und nach rechts aufgenommen.
