\section{Diskussion}
\label{sec:Diskussion}
  Die Mikroskopdaten sind entsprechend der Messmethode ungenau, aber eine notwendige Orientierung zum finden der Fitparameter.
  Der Rahmen auf dem Bildschirm war recht dick und ließ keine genaue Messung zu. Zudem ist der digitale Zoom des
  Mikroskops nur sehr grob einstellbar.

  Das Fitten der Daten gestaltete sich äußerst aufwendig; der zu fittende Sinus Cardinalis stellt ein großes Problem dar,
	da die zu minimierende Funktion der Summe der Abweichungsquadrate stark oszilliert und auf kleinen Bereichen viele
	lokale Minima besitzt, was eine Verwendung von Gradient-orientierten Fittern beinahe ausschließt, da diese hier
	nicht robust genug sind.
	Die Autoren benutzten schlussendlich eine modifizierte Version des \emph{Simplex}-Algorithmus in Tandem mit
	\emph{Mathematicas NonlinearModelFit}, die in der Kürze der Zeit ein noch annehmbares Ergebnis lieferten. Dabei wurde
	insbesondere darauf geachtet, die Frequenz (des Sinus Cardinalis) der Messdaten möglichst gut zu fitten, da diese für die Spaltbreite bestimmend
	ist.

	Die Messdaten zu den ersten drei Spalten sind gut genug, um die Spaltbreite zu bestimmen. Sie haben etwa 25~\% Abweichung
	zu den mikroskopierten Werten. Es gibt keine Anhaltspunkte, welches der beiden Messverfahren genauer ist.

	Die Messdaten des Doppelspalts weichen zu stark von dem nach der Theorie erwarteten Muster ab. Der hindurchgelegte Fit
	weicht zu stark ab, um aussagekräftig zu sein. Die Messdaten haben mehr den Charakter zwei weit voneinander entfernter
	Einzelspalte. Schlussendlich konnte nicht erklärt werden, warum der zu erwartende Peak in der Mitte des Diagramms ausbleibt.


\subsection{Messdaten} % (fold)
\label{sub:messdaten}

\begin{table}
\footnotesize
	\centering
  \caption{Messdaten zu Spalt 1.}
  \label{tab:fit}
  \begin{adjustbox}{center}
		\tabcolsep=0.11cm
  \begin{tabular}{
      S
      S
      S}
   \toprule
   \multicolumn{1}{c}{Winkel in mrad} &
   \multicolumn{1}{c}{Strom in \si{\mu\ampere}} &
   \multicolumn{1}{c}{Strom $\sigma$ in \si{\mu\ampere}} \\
   \midrule
      \primitiveinput{../table/single_slit1.tex}
   \bottomrule
  \end{tabular}
  \end{adjustbox}
\end{table}

\begin{table}
\footnotesize
	\centering
  \caption{Messdaten zu Spalt 2.}
  \label{tab:fit}
  \begin{adjustbox}{center}
		\tabcolsep=0.11cm
  \begin{tabular}{
      S
      S
      S}
   \toprule
   \multicolumn{1}{c}{Winkel in mrad} &
   \multicolumn{1}{c}{Strom in \si{\mu\ampere}} &
   \multicolumn{1}{c}{Strom $\sigma$ in \si{\mu\ampere}} \\
   \midrule
      \primitiveinput{../table/single_slit2.tex}
   \bottomrule
  \end{tabular}
  \end{adjustbox}
\end{table}

\begin{table}
\footnotesize
	\centering
  \caption{Messdaten zu Spalt 3.}
  \label{tab:fit}
  \begin{adjustbox}{center}
		\tabcolsep=0.11cm
  \begin{tabular}{
      S
      S
      S}
   \toprule
   \multicolumn{1}{c}{Winkel in mrad} &
   \multicolumn{1}{c}{Strom in \si{\mu\ampere}} &
   \multicolumn{1}{c}{Strom $\sigma$ in \si{\mu\ampere}} \\
   \midrule
      \primitiveinput{../table/single_slit3.tex}
   \bottomrule
  \end{tabular}
  \end{adjustbox}
\end{table}

\begin{table}
\footnotesize
	\centering
  \caption{Messdaten zu Spalt 4 (Doppelspalt).}
  \label{tab:fit}
  \begin{adjustbox}{center}
		\tabcolsep=0.11cm
  \begin{tabular}{
      S
      S
      S}
   \toprule
   \multicolumn{1}{c}{Winkel in mrad} &
   \multicolumn{1}{c}{Strom in \si{\mu\ampere}} &
   \multicolumn{1}{c}{Strom $\sigma$ in \si{\mu\ampere}} \\
   \midrule
      \primitiveinput{../table/single_slit4.tex}
   \bottomrule
  \end{tabular}
  \end{adjustbox}
\end{table}


% subsection messdaten (end)

\newpage
\section{Literaturangabe}
\label{sec:Literatur}

Bilder und Daten aus dem Skript zu \emph{Beugung am Spalt}, Versuch 406, TU Dortmund, abrufbar auf:\\
\url{http://129.217.224.2/HOMEPrismadispersionPAGE/PHYSIKER/BACHELOR/AP/SKRIPT/V406.pdf}\\(Stand 30.04.16)\par

Bild zu \emph{Intensitätsverteilung am Doppelspalt}, By Klaus-Dieter Keller - Own work, created with SciDAVisThis vector image was created with Inkscape., CC BY 3.0,
abrufbar auf: \\
\url{https://commons.wikimedia.org/w/index.php?curid=24952635 }\\ (Stand 30.04.16)

