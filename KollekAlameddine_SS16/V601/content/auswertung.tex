\section{Auswertung}
\label{sec:Auswertung}

% % Examples
% \begin{equation}
%   U(t) = a \sin(b t + c) + d
% \end{equation}
%
% \begin{align}
%   a &= \input{build/a.tex} \\
%   b &= \input{build/b.tex} \\
%   c &= \input{build/c.tex} \\
%   d &= \input{build/d.tex} .
% \end{align}
% Die Messdaten und das Ergebnis des Fits sind in Abbildung~\ref{fig:plot} geplottet.
%
% %Tabelle mit Messdaten
% \begin{table}
%   \centering
%   \caption{Messdaten.}
%   \label{tab:data}
%   \sisetup{parse-numbers=false}
%   \begin{tabular}{
% % format 1.3 bedeutet eine Stelle vorm Komma, 3 danach
%     S[table-format=1.3]
%     S[table-format=-1.2]
%     @{${}\pm{}$}
%     S[table-format=1.2]
%     @{\hspace*{3em}\hspace*{\tabcolsep}}
%     S[table-format=1.3]
%     S[table-format=-1.2]
%     @{${}\pm{}$}
%     S[table-format=1.2]
%   }
%     \toprule
%     {$t \:/\: \si{\milli\second}$} & \multicolumn{2}{c}{$U \:/\: \si{\kilo\volt}$\hspace*{3em}} &
%     {$t \:/\: \si{\milli\second}$} & \multicolumn{2}{c}{$U \:/\: \si{\kilo\volt}$} \\
%     \midrule
%     1.7 & 10 \\
2.3 & 20 \\
3.5 & 30 \\
4.4 & 40 \\

%     \bottomrule
%   \end{tabular}
% \end{table}
%
% % Standard Plot
% \begin{figure}
%   \centering
%   \includegraphics{build/plot.pdf}
%   \caption{Messdaten und Fitergebnis.}
%   \label{fig:plot}
% \end{figure}
%
% 2x2 Plot
% \begin{figure*}
%     \centering
%     \begin{subfigure}[b]{0.475\textwidth}
%         \centering
%         \includegraphics[width=\textwidth]{Abbildungen/Schaltung1.pdf}
%         \caption[]%
%         {{\small Schaltung 1.}}
%         \label{fig:Schaltung1}
%     \end{subfigure}
%     \hfill
%     \begin{subfigure}[b]{0.475\textwidth}
%         \centering
%         \includegraphics[width=\textwidth]{Abbildungen/Schaltung2.pdf}
%         \caption[]%
%         {{\small Schaltung 2.}}
%         \label{fig:Schaltung2}
%     \end{subfigure}
%     \vskip\baselineskip
%     \begin{subfigure}[b]{0.475\textwidth}
%         \centering
%         \includegraphics[width=\textwidth]{Abbildungen/Schaltung4.pdf}    % Zahlen vertauscht ... -.-
%         \caption[]%
%         {{\small Schaltung 3.}}
%         \label{fig:Schaltung3}
%     \end{subfigure}
%     \quad
%     \begin{subfigure}[b]{0.475\textwidth}
%         \centering
%         \includegraphics[width=\textwidth]{Abbildungen/Schaltung3.pdf}
%         \caption[]%
%         {{\small Schaltung 4.}}
%         \label{fig:Schaltung4}
%     \end{subfigure}
%     \caption[]
%     {Ersatzschaltbilder der verschiedenen Teilaufgaben.}
%     \label{fig:Schaltungen}
% \end{figure*}

\subsection{Bestimmung der freien Weglänge}
Zunächst werden die Sättigungsdampfdrücke $p_{\text{sätt}}$ aus Formel \eqref{eqn:8} sowie die mittleren Weglängen der Elektronen aus Formel \eqref{eqn:7} für die verschiedenen Temperaturen, bei denen die Experimente durchgeführt werden, bestimmt.
Die Ergebnisse sind in Tabelle \ref{tab:0} angegeben.
\input{build/Tabelle_0_texformat.tex}
Bei der hier verwendeten Apparatur beträgt der Abstand zwischen Kathode und Beschleunigungselektrode etwa $\SI{1}{\centi\metre}$.
Dies bedeutet, dass die mittlere Weglänge im Mikrometerbereich liegen sollte.
Folglich können gute Franck-Hertz-Kurven bei einer Temperatur von $\SI{400}{\kelvin}$ bis $\SI{450}{\kelvin}$ aufgenommen werden.

\subsection{Integrale Energieverteilung}
\label{sec:ie}
Wie in der Durchführung beschrieben wird eine feste Beschleunigungsspannung von $U_b = \SI{11}{\volt}$ gewählt, und der Auffängerstrom gemessen.
Um nun die integrale Energieverteilung beschreiben zu können, werden mehrere Spannungswerte $U_a$ gegen den dazugehörigen Wert
\begin{equation}
  \increment I_a \coloneq I_a(U_a) - I_a(U_a + \increment U_a)
\end{equation}
abgetragen.
Dementsprechend beschreibt dieser Zusammenhang nach der Formel \eqref{eqn:3} die Energieverteilung, da eine hohe Änderung von $I_a$ einer hohen Anzahl von Elektronen entspricht, die ebendiese kinetische Energie besitzen und nun durch die Bremsspannung aufgehalten werden.
Zur Bestimmung der Spannungsdifferenzen werden die Werte
\begin{align*}
  \increment U_{a,1} = \SI{0,394}{\volt} \\
  \increment U_{a,2} = \SI{0.292}{\volt}
\end{align*}
verwendet.
Die Ergebnisse sind in den Tabellen \ref{tab:1} und \ref{tab:2} angegeben sowie in den Abbildungen \ref{fig:plot1} und \ref{fig:plot2} graphisch dargestellt.
Auf eine Betrachtung der Messfehler wird hier verzichtet, da die Ablesefehler aus dem Diagramm einen größeren Faktor darstellen.

\input{build/Tabelle_a1_texformat.tex}
\input{build/Tabelle_a2_texformat.tex}

\begin{figure}[H]
  \centering
  \includegraphics{build/aufgabenteil_a_plot.pdf}
  \caption{Messdaten für die Integrale Energieverteilung.}
  \label{fig:plot1}
\end{figure}

\begin{figure}[H]
  \centering
  \includegraphics{build/aufgabenteil_a_plot_2.pdf}
  \caption{Messdaten für die Integrale Energieverteilung.}
  \label{fig:plot2}
\end{figure}

Für die erste Messung ist auffällig, dass ein Peak bei ca. $U_{\text{grenz}} = \SI{7.9}{\volt}$ existiert.
Da die Beschleunigungsspannung größer als diese effektive Elektronenenergie nach Durchlaufen des elektrischen Feldes ist, kann man von einem sich hier auswirkenden Kontaktpotential ausgehen.
Dieses beträgt dementsprechend ca. $K = \SI{3.1}{\volt}$.\\
%In der zweiten Messung erweist sich eine konkrete Zuordnung eines Peaks als schwieriger.
%Dies liegt vermutlich an der Tatsache, dass deutlich mehr inelastische Stöße stattfinden, da die mittlere Weglänge im Vergleich zur Messung bei Zimmertemperatur deutlich geringer ist.
%Ein theoretisch zu erwartender bei
%\begin{equation}
%  U = U_b - K - U_{\text{Hg}} = \SI{2.8}{\volt}
%\end{equation}
%ist leicht zu sehen.  %hier dann konkret werden wenn U_hg bei Teil b berechnet wurde. hab ich
In der zweiten Messung ist kein Peak mehr zusehen, da dort inelastische Stöße auftreten, wodurch die Elektronen nicht mehr genug Energie haben um die Auffängerelektrode zu erreichen.


\subsection{Interpretation der Franck-Hertz Kurven}
Die aufgenommene Franck-Hertz-Kurve bei $T = \SI{451.15}{\kelvin}$ ist in Abbildung \ref{fig:abb1} wiedergegeben.

\begin{figure}[H]
  \centering
  \includegraphics[height = 10cm]{messdaten/daten-4.jpg}
  \caption{Franck-Hertz-Kurve bei $T = \SI{451.15}{\kelvin}$.}
  \label{fig:abb1}
\end{figure}

Die ablesbaren Maxima befinden sich bei
\begin{align*}
  U_1 &= \input{build/b_max1.tex} \\
  U_2 &= \input{build/b_max2.tex} \\
  U_3 &= \input{build/b_max3.tex} \\
  U_4 &= \input{build/b_max4.tex}.
\end{align*}
Der gemittelte Abstand der Maxima beträgt entsprechend der Mittelwertsrechnung (Fehler des Mittelwertes)
\begin{align*}
  \Delta U &= \input{build/b_U_max_delta.tex}.
\end{align*}
Nach der Formel (\ref{eqn:2}) beträgt die Wellenlänge $\lambda$ der emittierten Strahlung (Gaußsche Fehlerfortpflanzung)
\begin{align*}
  \lambda &= \input{build/b_wellenlaenge.tex}.
\end{align*}

\subsection{Bestimmung der Ionisationsenergie}
Die Messkurve bezüglich der Untersuchung der Ionisationsenergie der Quecksilberatome ist in Abbildung \ref{fig:abb2} dargestellt.

\begin{figure}[H]
  \centering
  \includegraphics[height = 10cm]{messdaten/daten-5.jpg}
  \caption{Ionisationskurve von Quecksilber aufgenommen bei $T = \SI{309.75}{\kelvin}$.}
  \label{fig:abb2}
\end{figure}

Für die Position des Peaks, welcher bei $U = \SI{14.1}{\volt}$ zu sehen ist, gilt
\begin{equation}
  U_{\text{peak}} = U_{\text{Ion}} + K.
\end{equation}

Mit dem in Kapitel \ref{sec:ie} bestimmten Wert für $K$ folgt somit, dass die Ionisationsenergie von Quecksilber etwa
\begin{align*}
  E_{\text{Ion}} &= \input{build/c_ion.tex}
\end{align*}
beträgt.
